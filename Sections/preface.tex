% vim: ts=4 sts=4 sw=4 et tw=75
\chapter{前言}
\label{chap:preface}
\marginpar{1}
在计算机发展的早期,系统资源非常有限,开发人员必须想尽办法优化他们的应用程序,文本编辑器也是如此。Vim是当时最流行的编辑器之一,因为系统资源方面的限制,它被优化得近乎完美。\\

从那以后,计算机技术得到了快速的发展,虽然系统资源已经没有以前那么紧张了,但是Vim仍然坚持最初的原则。\\

乍看起来,Vim好像并没有什么值得称道的地方,但是,如果你能透过它简单的用户界面看到其本质,就会明白为什么到了如今这个年代,Vim仍然是众多用户最喜爱的编辑器。\\

它几乎囊括了你想要的任何一个功能,即使有所遗漏,也可以通过插件或脚本来实现。正是由于其出色的灵活性,使得它成为了众多任务的理想工具,以及世界上最先进的编辑器之一。\\

每天都有大量的新用户加入Vim社区,并开始用Vim处理他们的日常工作.虽然有时候使用起来比较复杂,但是与其他编辑器相比,人们还是更愿意选择Vim,本书就是为这些用户而写。\\

通过阅读本书,用户可以更加得心应手地使用Vim,从而提高工作效率。他们得到的不仅仅是一个优化的编辑器,还有优化的工作流程。本书帮助用户更加自如地使用Vim,从把它当作一个简单的文本编辑器开始,一直到其他日常工作,Vim都可以胜任。\\

祝你阅读愉快!

\section*{本书主要讨论什么}
\marginpar{2}
第\ref{chap:getting_started_with_vim}章: 开始,介绍Vim和它的几个比较有名的亲戚,并简要介绍它们和 vi 的关系及其历史。\\

第\ref{chap:personalizing_vim}章: 定制Vim,介绍如何配置Vim,使得它更符合用户的个人需求。主要介绍如何修改字体,配色方案,状态行,菜单,与工具条。\\

第\ref{chap:better_navigation}章: 快速导航,介绍一些在多个文件中快速导航的方法,通过这些方法,读者可以轻松地在多个文件之间穿梭。\\

第\ref{chap:production_boosters}章: 助推器,介绍Vim的若干特性,描述了模版,自动补全,拆叠,会话,和寄存器的使用方法。\\

第\ref{chap:advanced_formatting}章: 格式化进阶,讨论如何对文本和代码进行格式化。它还介绍了如何借助外部工具,使得Vim更加完美。\\

第\ref{chap:basic_vim_scripting}章:Vim脚本基础,这一章是为那些想通过脚本来扩展Vim功能的人而准备的,它介绍了脚本的基础知识,经过这一章的学习,读者应该能够写出一些简单的脚本。\\

第\ref{chap:extended_vim_scripting}:Vim脚本进阶,在第\ref{chap:basic_vim_scripting}章的基础上,再继续讲一些高级的脚本知识,包括如何在Vim脚本中使用外部的脚本语言。\\

附录 \ref{chap:vim_can_do_everything}: 无所不能的Vim,提供了一张游戏列表,它们都是用Vim脚本开发的,这一章还简单讨论了如何使用脚本实现聊天和邮件工具,另外还介绍了如何将Vim作为一个IDE使用。\\

附录 \ref{chap:vim_configuration_alternatives}:Vim配置管理,展示如何通过在线副本来更好地管理和获取Vim配置文件。\\

\section*{为了阅读本书,你还需要什么}
最近十年,Vim已经发展成一个功能非常丰富的编辑器,这同时意味着最新版的某些功能,旧版可能并不支持。\\

Vim 已经移植到了多种平台中,但并不是所有的功能对任意一种平台来说都是可用的。这主要是因为有些功能使用了和操作系统密切相关的特性,而这些特性在其他平台中可能并不提供。\\

\marginpar{3}
本书只关注Vim使用最广泛的两种平台:Linux与Microsoft Windows。由于Linux符合Unix标准,所以同样的结论在其他类Unix系统中仍然成立。\\

\begin{warning}
读者可以在 \url{www.vim.org} 上找到最新版的Vim源代码与二进制包。如果读者使用的是Linux,那么系统中很可能已经安装了Vim,而且是默认的编辑器。\\
\end{warning}

\section*{本书的目标读者}
\label{sec:who_this_book_is_for}
如果读者已经是一位Vim用户,而且想要学习更高级的技巧,那么这本书就是为你而写的。本书致力于帮助读者从一位Vim中级用户,上升为高级用户。\\

\section*{Vim 新手?}
\label{sec:new_to_vim}
虽然本书假设读者已经具备了Vim的基本知识,但即使是新手,本书也有一定的参考价值。\\

如果读者担心自己所掌握的Vim 知识不足以阅读此书,那么我建议你可以先看一下Vimtutor,在安装Vim时,会顺带安装这个Vim入门教程。\\

运行Vimtutor,然后跟着它学习Vim的基本使用方法,这大概需要30分钟。\\

Vimtutor支持多国语言,在启动程序时,可以通过指定国家代码来选择一种语言,国家代码由2个字母组成,比如ca,cs,de,el,eo,es,fr,hr,hu,it,ja,no,ro,和zh。例如,如果想要阅读德语版的Vimtutor,只需要在命令行执行\texttt{vimtutor de}。\\

\section*{排版约定}
\label{sec:conventions}
本书用不同风格的文本来表示不同种类的信息,这里有一些例子。\\

\marginpar{4}
包含在正文中的代码会这样显示:``键入 \texttt{:help 'statusline'},就可以打开Vim帮助系统,查看与状态行有关的全部帮助信息。''\\

代码块则是:
\begin{vimcode}
function! InfoGuiTooltip()
    "get window count
    let wincount = tabpagewinnr(tabpagenr(),'$')
    let bufferlist=''
    "get name of active buffers in windows
    for i in tabpagebuflist()
        let bufferlist .= '['.fnamemodify(bufname(i),':t').'] '
    endfor
    return bufname($).' windows: '.wincount.' ' .bufferlist ' '
endfunction
\end{vimcode}

命令模式的输入或输出写成:\\
\begin{vimcode}
:amenu icon=/path/to/icon/myicon.png ToolBar.Bufferlist :buffers<cr>
\end{vimcode}

\textbf{新的术语} \footnote{在中文版中并没有这样做 --- 译者注} 或\textbf{重点内容}加粗显示。读者在屏幕上看到的内容,比如菜单或对话框中的信息,会显示成:``这个命令在所有行的开头查找单词Error (用脱字符 \textasciicircum 标记)''。\\

\begin{warning}
这种文本框用来显示警告或重要的提示。\\
\end{warning}

\begin{tips}
这种文本框用来显示技巧和决窍。\\
\end{tips}

\section*{读者反馈}
\label{sec:reader_feedback}
对于读者的反馈我们总是非常欢迎。请把你关于这本书的想法告诉我们---不管是优点还是缺点,读者的反馈永远是我们不断进步的源泉。\\

反馈信息请发送到 \email{feedback@packtpub.com},并在邮件的主题中写上相关的书名。\\

\marginpar{5}
如果读者想向我们推荐书籍,请登陆\url{www.packtpub.com},在表单\texttt{SUGGEST A TITLE} 中填写相关的信息,或者发送邮件到\email{suggest@packtpub.com}。\\

如果你对某个主题很有研究,并且对写作也很感兴趣,请阅读\url{www.packtpub.com/authors} 的相关内容。\\

\section*{客户支持}
\label{sec:customer_support}既然读者买了 Packt 出版的书籍,那么我们会尽最大的努力让你觉得物有所值。\\
\begin{warning}
    下载书中的示例源代码:\\

    登陆 \url{http://www.packtpub.com/files/code/0509_Code.zip},下载书中的示例源代码,下载的文件中含有使用方法。\\
\end{warning}

\section*{勘误}
\label{sec:errata}

我们已经尽了最大的努力来保证内容的正确性,如果读者在书中发现了错误 --- 这个错误可能出现在正文中,也可能出现在代码中 --- 请把错误反馈给我们。通过反馈错误,可以帮助其他读者更顺利地阅读,也可以帮助我们提升后续版本的质量。无论读者发现了什么类型的错误,请登陆到\url{http://www.packtpub.com/support},选择对应的书籍,点击 \texttt{let us
know},然后输入相关的勘误信息。一旦勘误通过了校验,你的提交信息就会被接受,勘误也会上传到网页中,或者加到书籍勘误列表,这个列表可以在书籍的\texttt{Errata}部分找到。登陆\url{http://www.packtpub.com/support},选择对应的书籍,读者就可以找到该书所有的已知勘误。\\

\section*{版权声明}
\label{sec:piracy}

因特网物品的版权一直是一个让人头疼的问题。Packt 对版权和授权的保护向来非常重视,态度也很坚决。如果读者怀疑有人在因特网上非法复制我们的作品 --- 无论是什么形式 --- 请把相关的网址或网站名称发给我们(\email{copyright@packtpub.com}),这样我们才能快速地追回损失。希望读者能和我们一起尊重作者的辛勤付出,这样我们才能继续为你奉献精彩的作品。\\
\marginpar{6}

\section*{答疑}
\label{sec:questions}
无论你对书籍有什么疑问,都可以联系我们:\email{questions@packtpub.com},我们会尽最大的努力来回答你的问题。\\