% vim: ts=4 sts=4 sw=4 et tw=75
% preamble here.

\documentclass[nofonts, oneside]{ctexbook}

\usepackage{geometry}
\usepackage{fontspec}
\usepackage{xeCJK}
\usepackage{amssymb}
\usepackage{hyperref}
% for number of footnote
\usepackage{pifont}
% 一页结束时, 脚注编号清零
\usepackage[perpage]{footmisc}
% verbatim and endverbatim
\usepackage{graphicx}
\usepackage{listings}
\usepackage{multicol}
\usepackage{tocloft}
\usepackage{graphics,graphicx,pdfpages}

% for fixing "Unable to load picture or PDF file 'bc-book.mps'"
\DeclareGraphicsRule{.mps}{eps}{*}{}

% 脚注编号带圈
\renewcommand\thefootnote{\ding{\numexpr171+\value{footnote}}}

% from package geometry
% 为边注加边框
\let\oldmarginpar=\marginpar
\renewcommand\marginpar[1]{%
    \oldmarginpar{\framebox{#1}}%
}
\geometry{%
	margin=1cm,
	marginparsep = 0.5cm,
	marginparwidth=1cm,
	top = 3cm,
	bottom = 3cm,
	outer = 3cm,
	inner = 3cm
}

% from package hyperref
\hypersetup{
    bookmarksnumbered = true,
    pdftitle = {Hacking Vim 7.2},
    pdfcreator = {https://m-mono.github.io},
    pdfauthor = {Kim Schulz},
    pdfsubject = {深入 Vim 7.2},
    colorlinks = false,
    pdfborder = 0 0 0,
    pdfkeywords = {不知道是谁 说过一句话:说世界上的程序员分为三种,一种是使用Vim的,还有一种使用Emacs的,剩下的都是第三种。由此可见Vim的强大程度,可以说,这是一款非常适合程序员使用的编辑器。Vim的设计哲学是如果你的工作只需要做一次,那没问题,怎么搞都行,如果你的工作是要重复的完成某些工作,则Vim总可以帮你找到更少的按键方式来实现相同的目标,为此,Vim设计了多种模式,常见的模式包括命令模式,编辑模式,命令模式等,这个也是和其他编辑器一个很不相同的地方,可能有些已经有其他编辑器习惯的人有点不太习惯的地方,但是如果将模式熟悉了就可以发现在不同的模式之间切换是很平常的事情,而且带来的好处是更少的按键和更快的输入。}
}




%\setCJKfamilyfont{heiti}{FandolHei}
% 章节格式
%%\CTEXsetup[format={\CJKfamily{heiti}\large\upshape}]{subsection}
%%\CTEXsetup[numberformat={\CJKfamily{heiti}\large\upshape\bfseries}]{subsection}
%%\CTEXsetup[format={\CJKfamily{heiti}}]{subsubsection}
% 目录格式
%\renewcommand\cftchapfont{\CJKfamily{heiti}}
\settowidth\cftchapnumwidth{第几十几章} % 最宽的可能编号
\renewcommand\cftchapaftersnumb{\hspace{2.2em}} % 额外间距

% from package fontspec and xeCJK
\setCJKmainfont[Path=Fonts/]{Inziu-Iosevka-SC-Regular.ttf}
\setCJKsansfont[Path=Fonts/]{Inziu-Iosevka-SC-Regular.ttf}
\setCJKmonofont[Scale=0.9,Path=Fonts/]{Inziu-Iosevka-SC-Regular.ttf}
\setmainfont[Path=Fonts/]{Inziu-Iosevka-SC-Regular.ttf}
\setsansfont[Path=Fonts/]{Inziu-Iosevka-SC-Regular.ttf}
% "Mapping={}" make quote symbol straight
\setmonofont[Mapping={},Path=Fonts/]{SourceCodePro-ExtraLight.ttf}

\newcommand\vi{\texttt{vi}}
\newcommand\vim{\texttt{vim}}

% TODO: warning 与 tips 环境有一对中括号, 而且高度随内容而改变.
\newenvironment{warning}%
{
    \begin{quotation}
        \parbox[c]{4em}{\includegraphics[scale=1.8]{./images/bc-book.mps}}
    \begin{minipage}{33em}
}
{
    \end{minipage}
    \end{quotation}
}

\newenvironment{tips}%
{
    \begin{quotation}
        \parbox[c]{4em}{\includegraphics[scale=1.8]{./images/bc-lampe.mps}}
    \begin{minipage}{33em}
}
{
    \end{minipage}
    \end{quotation}
}

\newcommand\file[1]{\texttt{#1}}
\newcommand\newterm[1]{#1}
\newcommand\email[1]{\href{mailto:#1}{#1}}
\newcommand\key[1]{\textit{#1}}

\newenvironment{vimcmdform}
    {\list{}{\leftmargin=2em\rightmargin=0em}\item[]}
    {\endlist}

\lstnewenvironment{vimcode}
    {\lstset{basicstyle=\ttfamily,
        xleftmargin=2em,
        breaklines=true,
        tabsize=4}}
    {}

\title{Hacking Vim 7.2 / 深入 Vim 7.2}
\author{Kim Schulz}