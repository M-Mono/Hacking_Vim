% vim: ts=4 sts=4 sw=4 tw=80
\chapter{Vim 配置管理}
\label{chap:vim_configuration_alternatives}
\marginpar{215}

在第 \ref{chap:personalizing_vim} 章, 我们介绍了 Vim 的主要配置文件, 随着
阅读的深入, 我们不断地往 \texttt{vimrc} 中添加新的配置信息, 最终, 配置文件
可能会变得非常混乱, 难以管理.

在这个附录中, 我们将介绍一些组织 \texttt{vimrc} 的方法, 从小技巧一直到整个
配置系统.

最后, 我们将会介绍如何在多个不同的计算机中使用同一个 \texttt{vimrc} 文件,
方法是在网络中保存一份副本.

\section{保持 vimrc 整洁的技巧}
\label{sec:tips_for_keeping_your_vimrc_file_clean}

\texttt{vimrc} 是用户设置 Vim 的核心文件, 如果没有它, 就只能使用系统中原有
的设置, 因此我们要时刻保持 \texttt{vimrc} 的整洁, 并及时更新, 只有这样, 用
户才能时刻知道文件中包含了哪些内容. 有时候, 用户可能没办法在文件中找到自己
想要的内容, 笔者就曾经遇到过这种情况, 当时我的 \texttt{vimrc} 超过了 2000 行,
到那时我才意识到整洁的必要性. 保持 \texttt{vimrc} 整洁并组织良好的技巧有:
\marginpar{216}
