% vim: ts=4 sts=4 sw=4 et tw=75
\chapter{更好的导航}
\label{chap:better_navigation}
\marginpar{53}
同时应付多个文件可能是一件非常麻烦的事, 有时候, 用户可能会花更多的时间来
定位文件, 而不是编辑.

Vim 的处事哲学是不浪费用户的宝贵时间, 所以它提供了许多用于定位文件的方法.

在这一章, 我们将会学习到 Vim 如何帮助我们在多个文件中导航, 无论此时是在
处理 1 个文件, 还是 50 个文件. 其中的某些方法使用标记, 以便于稍后返回到该
区域, 还有些方法使用搜索来定位目标.

这一章包含的内容有:
\begin{itemize}
    \item 在单个文件中更快地导航
    \item 在 Vim 帮助系统中更快地导航
    \item 在多个缓冲区更快地导航
    \item 使用 Vim 文件浏览器, 以便更快地搜索文件
    \item 文件内搜索
    \item 使用 \texttt{vimgrep} 在多个文件或多个缓冲区中搜索
    \item 使用标记作为导航的工具
    \item 使用符号来得到更好的概览
\end{itemize}

学习后这一章之后, 用户的导航速度将会有质的提升, 在搜索文件时也不会再遇到
什么问题.
\marginpar{54}
\section{在文件内更快地导航}
\label{sec:faster_navigation_in_a_file}

有时候, 即使是一件最简单的工作 --- 比如在一个单独的文件中导航 --- 也有优化
的空间. Vim 提供了几种在文件内导航的方法, 这些方法可以根据文件的内容和组织
结构而加以调整. 其中有些方法非常简单, 而另外一些则比较复杂.

\subsection{基于上下文的导航}
\label{subsec:context_aware_navigation}

在大部分情况下, 正在编辑的文件是有结构的. 如果是普通的文本文件, 那么文件的结
构可以是段落, 语句, 单词, 在另外的一些场合中, 还有可能是函数, 代码块和代码
行.

Vim 支持根据文件的结构, 在文件中跳转. 它还提供了一些按键绑定, 从而可以更
方便地跳转到某个特定的位置.

让我们来看一些例子:
\begin{itemize}
    \item 在普通的文本文件中移动
    \item 在代码文件中移动
\end{itemize}

\subsubsection{在普通的文本文件中移动}
\label{subsubsec:moving_around_within_a_text_file}

假设用户正在编辑一个普通文件文件, 此时光标正停留在一个句子的中部, 而用户
突然意识到自己忘了把本段的第一个字母大写. 虽然用户可以通过方向键, 或
\key{h}, \key{j}, \key{k}, \key{l}, 把光标移到段落的首字母. 然而, 在普通 
模式下, 直接按下面这个按键可以得到更好的效果:
\begin{vimcmd}
{
\end{vimcmd}

按完这个按键之后, 光标已经停在了段落的开头, 或者是段落正上方的空行 (如果 
有的话). 现在, 用户可以通过按下 \key{Esc} 进入到普通模式, 再按 \key{\{},
把光标移到段落的开始. 与此类似, 用户只要按下和 \key{\{} 相对的按键, 即:
\begin{vimcmd}
}
\end{vimcmd}
就可以把光标移到段落的末尾.

也许用户并不是在段落的末尾工作, 而是在修改段落中的某些错误. Vim 可以记住
用户之前修改过的地方 (实际上, Vim 可以记住最近 999 个被修改过的地方), 然
后, 用户想
