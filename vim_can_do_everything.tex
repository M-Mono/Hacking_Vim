% vim: ts=4 sts=4 sw=4 tw=80
\chapter{无所不能的 Vim}
\label{chap:vim_can_do_everything}
\marginpar{201}

据说 Vim 可以做任何事情, 虽然这可能不是真的, 但 Vim 的确可以完成许多读者根本
想都不会想到的事情.

在这一节, 我们将会介绍一些原来需要通过 Vim 脚本, 或外部工具才能完成的的事情.

这一节讨论的内容涵盖了游戏, 邮件客户端, IRC 即时聊天, 与集成开发环境的设置.

\section{Vim 游戏}
\label{sec:vim_games}

虽然 Vim 只是一个文本编辑器, 但仍然有很多人开发了大量的脚本, 来让 Vim 完成
其他一些非编辑任务, 其中包括可以直接在 Vim 中玩的小游戏. 这些小游戏不是那些
类似于 ``20 个小问题'' 那样的文本类游戏, 而是带有图形界面的. 这些图形界面
并不完美, 因此它们是用 ASCII 码字符组成的界面, 但对于游戏来说已经足够了.
\marginpar{202}

\subsection{生命游戏}
\label{subsec:game_of_life}

第一个要介绍的游戏严格说来, 并不能算作游戏, 但是仍然值得说一下. 生命游戏 (
The Game of Life) 通常被人称为无玩家游戏, 因为这个游戏并不需要玩家参与, 只需
要静静地看着就行. 游戏是一个非常简单的人工智能, 用来模拟细胞的演化. 细胞要
遵守下面几条规则:
\begin{enumerate}
	\item 在一个活的细胞周围, 如果活细胞数少于两个, 则这个细胞会由于孤独而
		死去
	\item 在一个活的细胞周围, 如果活细胞数多于三个, 则这个细胞会由于拥挤而
		死去
	\item 在一个活的细胞周围, 如果活细胞数是两个或三个, 则这个细胞会继续存
		活下去
	\item 在一个死的细胞周围, 如果活细胞数是三个, 则这个细胞会复活
\end{enumerate}

1996 年, 一个自称为大胡子艾丽 (Eli the Bearded) 的人开发了一个 Vim 脚本, 该
脚本实现了生命游戏. 脚本的运行速度并不是非常快, 仅仅是为了阐明游戏的原理.
对于大多数人来说, 这个游戏非常的枯燥, 但是对于生命游戏迷来说, 这个实现非常
的游戏.

可以到下面这个网址下载生命游戏的 Vim 脚本:
\url{http://www.vanhemert.co.uk/vim/vimacros/life1.vim}.

\subsection{贪吃蛇}
\label{subsec:nibbles}

1986 年, 笔者拥有了第一台 PC, 这台 PC 上只有一个游戏可以玩 --- 贪吃蛇. 笔者
花了很长时间玩这个游戏, 游戏的内容是控制一条小蛇的爬行, 有很多个关卡, 在每个
关卡中, 小蛇都要吃完一定数量的方块后才能过关, 随着方块的增多, 小蛇身体的长度
也在加长. 规则是小蛇不能穿越边界, 也不能穿越自己的身体.

2004 年, Hari Krishna Dara 重新用 Vim 脚本实现了这个游戏. 虽然他的游戏只包含
了很少的关卡, 但是只要用户需要, 就可以添加关卡. 虽然游戏在运行时需要不停地打
印文本, 但是看起来非常得流畅.
\marginpar{203}

\subsection{推箱子}
\label{subsec:sokoban}

笔者比较喜欢智力游戏, 也很喜欢推箱子. 游戏玩起来非常简单, 但是游戏本身却可以
非常得难. 游戏的内容是让一个小人把箱子推到指定的位置上. 听起来很简单是吗?
游戏的规则是一次只能推一个箱子, 而且只能推, 不能拉, 因此, 如果把箱子推到角落
中, 那就再也推不动它了, 这时候玩家就不得不重新开始游戏.

2002 年, Mike Sharpe 用 Vim 脚本实现了这个游戏, 游戏的关卡设计来自以前的
Linux 游戏 XSokoban. Mike Sharpe 尽量保持了用户接口的简单, 但游戏玩起来仍然
十分有趣.

\begin{warning}
    XSokoban 游戏可以到这个网址下载:
    \url{http://www.cs.cornell.edu/andru/xsokoban.html}.
\end{warning}

游戏脚本的下载地址是 \url{http://www.vim.org/scripts/script.php?script_id=211}.
\marginpar{206}
