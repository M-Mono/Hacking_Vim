% vim: ts=4 sts=4 sw=4 et tw=80
\chapter{效率推进器}
\label{chap:production_boosters}

\marginpar{73}
在这一章我们将会看到, 即使是一些小小的改动, 也可以极大地促进 Vim 的工作效率.
有些技巧是由 Vim 的特性提供的, 另外一些则需要用户自已编写一些脚本.

无论你把 Vim 当成一个修改配置文件的小工具, 还是把它用作某个大型开发项目的主要
编辑器, 你都可以发现本章介绍的方法可以极大地促进 Vim 的使用效率.

这一章讨论的主题包括:
\begin{itemize}
    \item 使用模版文件的模版
    \item 使用缩写的模版
    \item 使用已知单词与 tag list 的自动补全
    \item 使用 omnicompletion 的自动补全
    \item 宏与宏录制
    \item 使用会话
    \item 使用会话的项目管理
    \item 寄存器与撤消分支
    \item 折叠
    \item 使用 \texttt{vimdiff} 分析差异
    \item 使用 \texttt{netrw} 来随时随地地打开文件
\end{itemize}

阅读完这一章之后, 用户使用 Vim 的工作效率应该可以提高好几个百分点.
\marginpar{74}

\section{使用模版}
\label{sec:using_templates}

无论编辑的是哪一种类型的文件, 当打开一个新文件时, 总有一些基础性的工作需要完成.
手动完成这些基础工作是一件非常乏味的事件, 更讨厌的是每次打开一个新文件时, 都
要重新再做一遍. 所以说干嘛要花这么多的时间, 来做一件使用模版就可以完成的事情?

在接下来的两节, 我们将会看到一些不同类型的模版. 其中一些模版特定于文件类型, 另
外一些则会使用用户的输入来触发小内容模版 (比如, 程序员经常用到的代码片断).

\subsection{使用模版文件}
\label{subsec:using_template_files}

每次打开一个新文件时, 用户做的第一件事经常是输入某些头部信息, 当然, 所要输入的
信息取决于文件的类型. 比较常见的例子包括:
\begin{itemize}
    \item 在新的 HTML 文件中添加基本结构 (\texttt{<html>}, \texttt{<head>},
        \texttt{<body>}).
    \item 在所有的 C 文件添加头部信息, 在文件 \texttt{main.c} 中添加
        \texttt{main()} 函数.
    \item 在 Java 文件中添加主类.
\end{itemize}
除了这些, 你应该还能想到其他一些例子.

那么, 我们怎么才能创建一个模版文件? 不妨让我们用 HTML 文件作为例子来进行讲解.
这种文件的结构是静态的, 因此非常适合用模版来处理. HTML 模版的内容是:
\begin{vimcmd}
<html>
    <head>
        <title></title>
            <meta name="generator" content="Vim" />
            <meta name="author" content="Kim Schulz" />
    </head>
    <body>
        <p>Content goes here...</p>
    </body>
</html>
\end{vimcmd}
我们在 \texttt{VIMHOME} 目录下创建一个新目录 \texttt{templates/}, 并把上面的
模版文件保存到这个目录中, 假设我们把模版文件命令为 \texttt{html.tpl}.
\marginpar{75}
