% vim: ts=4 sts=4 sw=4 et tw=80
\chapter{效率推进器}
\label{chap:production_boosters}

\marginpar{73}
在这一章, 我们将会看到, 即使是一些小小的改动, 也可以地促进 Vim 的工作效率. 有些
技巧是由 Vim 的特性提供的, 另外一些则需要用户自已编写一些脚本.

无论你把 Vim 当成一个修改配置文件的小工具, 还是把它用作某个大型开发项目的主要
编辑器, 你都可以发现本章介绍的方法可以极大地促进使用 Vim 的使用效率.

这一章讨论的主题包括:
\begin{itemize}
    \item 使用模版文件的模式
    \item 使用缩写的模版
    \item 使用已知单词与 tag list 的自动补全
    \item 使用 omnicompletion 的自动补全
    \item 宏与宏录制
    \item 使用会话
    \item 使用会话的项目管理
    \item 寄存器与撤消分支
    \item 折叠
    \item 使用 \texttt{vimdiff} 分析差异
    \item 使用 \texttt{netrw} 来随时随地地打开文件
\end{itemize}

阅读完这一章之后, 用户使用 Vim 的工作效率应该可以提高好几个百分点.
\marginpar{74}
