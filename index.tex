% vim: ts=4 sts=4 sw=4 tw=80
\chapter{索引}
\label{chap:index}

\marginpar{223}
索引中的页码指的是英文原版的页码, 与本书页边标注的页码一致.

\begin{multicols}{2}

\section*{符号}

\texttt{:diffsplit} 112

\texttt{:diffthis} 112

\texttt{:pop} 82

\texttt{:ptselect} 82

\texttt{:tag} 82

\texttt{:tnext} 82

\texttt{:tprev} 82

\texttt{:tselect} 82

\texttt{:vert diffsplit} 112

\texttt{<cWROD>} 61

\texttt{<Leader>} 178

\texttt{<Plug>} 178

\texttt{<silent>} 31

\texttt{<unique>} 178

\section*{A}

\texttt{AbbrAsk} 49

\texttt{amenu} 33

\texttt{argnum} 171

Aspell 40

autocompletion \par
  about (关于) 84 \par
  all-in-one completion (多合一补全) 89, 90 \par
  dictionary completion (字典补全) 85,86 \par
  known word completion (已知单词补全) 84,85 \par
  omnicompletion 86-88 \par

\texttt{autoindent} 130

\texttt{autoproto.vim} 84

\section*{B}

balloons 43

Berkeley Par, external formatting tools (外部格式化工具) 137

black hole register (黑洞寄存器) 102

\texttt{buffers} 32

\section*{C}

\texttt{calcValue} 81

charityware license (慈善授权) 15

cindent \par
  about (关于) 131 \par
  setup options (配置选项) 131 \par

code block formatting (代码块格式化) \par
  commands (命令) 133 \par

code, formatting (代码格式化) \par
  about (关于) 129 \par
  \texttt{autoindent} 130 \par
  cindent 131 \par
  code-block, formatting (代码块格式化) 132-134 \par
  \texttt{indentexpr} 132 \par
  pasted code, auto formatting (自动格式化粘贴的代码) 135 \par
  settings (设置) 130 \par
  \texttt{smartindent} 130 \par

color scheme (配色方案) 147

color scheme, Vim (Vim 的配色方案) \par
  changing (修改) 21 \par

configuration files (配置文件) \par
  \texttt{exrc} 19 \par
  \texttt{gvimrc} 19 \par
  types (配置文件的类型) 18 \par
  \texttt{vimrc} 18,19 \par

context-aware navigation (基于上下文的导航) \par
  about (关于) 54 \par
  code file, moving around within (在代码文件内导航) 54, 55 \par
  code file, moving in (在代码文件内导航) 56-58 \par

ctags command-line program (ctags 命令行程序) 80

\texttt{ctags.vim} 83

cURL 213

\texttt{cursorline} 38

\section*{D}

debugger commands (调试器命令) \par
  about (关于) 187 \par
  \texttt{cont} 187 \par
  \texttt{finish} 187 \par
  \texttt{interrupt} 187 \par
  \texttt{next} 187 \par
  \texttt{quit} 187 \par
  \texttt{step} 187 \par

debugging (调试) \par
  Vim scripts (Vim 脚本) 186-188 \par

drop registers (投递寄存器) 102

\section*{E}

editor area, Vim (Vim 的编辑器区域) \par
  abbreviations, using (使用缩写) 46-48 \par
  key bindings, modifying (修改按键绑定) 49, 51 \par
  line numbers, adding (添加行号) 39 \par
  personalizing (个性化) 37 \par
  spell check (拼写检查) 40-42 \par
  tooltip, adding (工具提示) 43-46 \par
  editor area, Vim visual cursor, adding (添加可视化光标) 37, 38 \par

Elvis \par
  about (关于) 10 \par
  features (特性) 10 \par

Emacs editor (Emacs 编辑器) 13

expression register (表达式寄存器) 103

\texttt{exrc} 19

external formatting tools (外部格式化工具) \par
  about (关于) 136 \par
  Berkeley Par 137 \par
  Indent 136 \par
  Tidy 138 \par
  using (使用方法) 136 \par

external interpreter (外部解释器) \par
  using in Vim scripting (在 Vim 脚本中使用) 194 \par

\section*{F}

file explorer (文件浏览器) 208

file navigation (文件导航) \par
  about (关于) 54 \par
  context-aware navigation (基于上下文的导航) 54 \par
  long line, navigating (长行导航) 59 \par

File-Type plugins group (文件类型插件组) 148

fold (折叠) \par
  about (关于) 107 \par
  \texttt{diff}, using to track changes (使用 \texttt{diff} 跟踪差异) 114 \par
  simple text file outlining (简单的文本文件提纲) 110, 111 \par
  types (类型) 107 \par
  using (使用方法) 107-109 \par
  \texttt{vimdiff}, using to track changes (使用 \texttt{vimdiff} 跟踪差异) 111 \par

\texttt{foldclose()} 45

fonts, Vim (Vim 字体) \par
  changing (修改) 20 \par

\texttt{for} loop (\texttt{for} 循环) 164, 165

\texttt{formatexpr} 123

formatting, Vim (Vim 格式化) \par
  code, formatting (代码格式化) 129 \par
  external formatting tools, using (使用外部格式化工具) 136 \par
  text, formatting (文本格式化) 121 \par

functions (函数) \par
  creating (创建) 168-170 \par
  variable argument list (可变参数列表) 170-172 \par

\section*{G}

Game of Life (生命游戏) 202

\texttt{get} 163

Global plugins group (全局插件组) 148

\texttt{guitablabel} 属性 36

\texttt{gvimrc} 19

\section*{H}

hacker (黑客) 15

hacking 15

\texttt{hasmapto()} 178

\texttt{helpgrep} 67

hidden markers (隐藏标记) \par
  about (关于) 71 \par
  marks, using (使用标记) 71 \par

\section*{I}

IDE (集成开发环境) 208

\texttt{indentexpr} 132

Indent, external formatting tools (外部格式化工具) 136

integrated compiler (集成的编译器) 207

integrated debugger (集成的调试器) 207

interpreter (解释器) 15
Ispell 40

\section*{J}

\texttt{join} 163

\section*{K}

key bindings, editor area (按键绑定) \par
  modifying (修改) 49, 51 \par

\texttt{keys()} 165

\texttt{keyvar} 165

\section*{L}

line numbers, editor area (行号) \par
  adding (添加) 39 \par

\texttt{loaded\_myscript} 176

\texttt{LoadTemplate} 77

\texttt{lookupfile.vim} 83

loops (循环) \par
  about (关于) 164 \par
  \texttt{for} loop (\texttt{for} 循环) 164 \par
  types (类型) 164 \par
  \texttt{while} loop (\texttt{while} 循环) 164 \par

\section*{M}

marcro recording (宏录制) \par
  about (关于) 90 \par
  using (使用方法) 90-92 \par

\texttt{map} 30, 163

\texttt{marks} (书签) 71

matching features, Vim (Vim 的匹配特性) 22

\texttt{menu} 30

MicroEmacs code (MicroEmacs 的代码) 13

Mines (扫雷) 204

MS Visual Studio\textregistered 207

Mutt 210

\texttt{mydict} dictionary (字典 \texttt{mydict}) 165

\texttt{MyIndenter()} 132

\section*{N}

named registers (命名寄存器) 101

navigation, in multiple buffers (在多个缓冲区中导航) 61

navigation, in Vim help (在 Vim 帮助系统中导航) 60

Nibbles (贪吃蛇) 202

numbered registers (编号的寄存器) 100

nvi \par
  about (关于) 10 \par
  features (特性) 11 \par

\section*{O}

omnicompletion 86-88

\section*{P}

Perl 195

personal hightlighting, Vim (Vim 的修改化高亮) \par
  about (关于) 22, 23 \par
  color character, marking (彩色字符标记) 24 \par
  errors, preventing (抑制错误) 26 \par
  tabs not used for indentation, marking (标记不是用作缩进的制表符) 25 \par

\texttt{PrintSum} 169

project browser (项目浏览器) 208

Python 196

\section*{R}

\texttt{range()} 164

read-only registers (只读寄存器) 101

registers (寄存器) \par
  about (关于) 98 \par
  drop registers (投递寄存器) 102 \par
  expression registers (表达式寄存器) 103 \par
  named registers (命名寄存器) 101 \par
  numbered registers (带编号的寄存器) 100 \par
  read-only registers (只读寄存器) 101 \par
  search pattern register (搜索模式寄存器) 102 \par
  selection register (选择寄存器) 102 \par
  small delete register (小删除寄存器) 100 \par
  unnamed register (匿名寄存器) 100 \par
  using (使用方法) 99 \par

remote files (远程文件) \par
  editing (编辑) 117, 118 \par
  working in (处理远程文件) 115-117 \par

Rubik's cube (魔方) 203

Ruby 198

\section*{S}

script (脚本) 15

script development (脚本开发) \par
  about (关于) 150, 151 \par
  script writing basic (脚本开发基础) 151 \par

scripting tips, Vim (Vim 脚本开发技巧) \par
  Gvim, using (使用 Gvim) 182 \par
  longer lines, printing (打印长行) 185 \par
  multiple operating system, working with (处理多个操作系统) 183 \par
  versions, of Vim (Vim 的版本) 183 \par

script writing basics (脚本开发基础) \par
  about (关于) 151 \par
  conditions (条件) 157, 158 \par
  dictionaries, working with (处理字典) 159-163 \par
  functions, creating (创建函数) 168 \par
  lists, working with (处理线性表) 159-163 \par
  loops (循环) 164 \par
  types (类型) 152 \par
  variables (变量) 153-157 \par

\texttt{scrollbind} 113

search pattern register (搜索模式寄存器) 102

search, Vim (Vim 搜索) \par
  examples (例子) 64, 65 \par
  help system, searching (在帮助系统中搜索) 67 \par
  searching, in current file (在当前文件中搜索) 64 \par
  searching, in multiple files (在多个文件中搜索) 65, 66 \par

selection registers (选择寄存器) 102

\texttt{sessionoptions} \par
  \texttt{blank} 96 \par
  \texttt{buffers} 96 \par
  \texttt{curdir} 96 \par
  \texttt{folds} 96 \par
  \texttt{globals} 96 \par
  \texttt{help} 96 \par
  \texttt{localoptions} 96 \par
  \texttt{options} 96 \par
  \texttt{resize} 96 \par
  \texttt{sesdir} 96 \par
  \texttt{slash} 96 \par
  \texttt{tabpages} 96 \par
  \texttt{unix} 96 \par
  \texttt{winpos} 96 \par
  \texttt{winsize} 96 \par

sessions (会话) \par
  \texttt{sessionoptions} 96 \par
  simple session usage (会话的简单用法) 93-95 \par
  using (使用方法) 93 \par
  using, as project manager (项目管理程序) 97, 98 \par

setup options, cindent (cindent 的设置选项) \par
  \texttt{cinkeys} 131 \par
  \texttt{cinoptions} 131 \par
  \texttt{cinwords} 131 \par

\texttt{ShortTabLine()} 35

\texttt{sign} 68

Single Unix Specification (SUS, 单一 Unix 规范) 9

small delete register (小删除寄存器) 100

\texttt{smartindent} 130

snipMate plugin (snipMate 插件) 79

snipMate system (snipMate 系统) 79

Sokoban (推箱子) 205

\texttt{sort()} 165

\texttt{spelllang} 40

\texttt{spellsuggest()} 45

\texttt{split} 163

status line, Vim (Vim 的状态行) 26, 28

STEVIE 9

\texttt{suffixadd} 62

\texttt{sum} 170

syntax coloring (语法高亮) 142, 143, 147

syntax-color scheme (语法高亮主题) 141

syntax regions (语法区) 143-146

\section*{T}

tabs, Vim (Vim 的标签页) \par
  modifying (修改) 33-37 \par

tag browser (Tag 浏览器) 208

tag list generators (Tag list 生成程序) \par
  about (关于) 80 \par
  Ctags 80 \par
  Hdrtags 80 \par
  Jtags 80 \par
  Ptags 80 \par
  Vtags 80 \par

tag lists \par
  about (关于) 80 \par
  taglist navigation (taglist 导航) 83 \par
  uses 83 \par
  using (使用方法) 80-82 \par

\texttt{taglist.vim} 83

templates (模版) \par
  abbreviations, using (使用缩写) 76, 77 \par
  about (关于) 74 \par
  snippets, with snipMate script (通过 snipMate 使用代码片断) 78, 79 \par
  template files, using (使用模版文件) 74, 75 \par

Tetris (俄罗斯方块) 206

text, formatting (文本格式化) \par
  about (关于) 121 \par
  headlines, marking (标题行标记) 125, 126 \par
  lists, creating (创建线性表) 127-129 \par
  text, aligning (文本对齐) 124 \par
  text, putting into paragraph (文本分段) 122, 123 \par

The Mail Suite (TMS, 邮件套装) 210

Tic-Tac-Toe (井字棋) 204

Tidy, external formatting tools (外部格式化工具 Tidy) 138

TwitVim 212

\section*{U}

undo branching (撤消分支) \par
  about (关于) 98 \par
  using (使用方法) 103-106 \par

unnamed register (匿名寄存器) 100

\section*{V}

variables (变量) \par
  about (关于) 153 \par
  dictionary (字典) 153 \par
  \texttt{funcref} 153 \par
  list (线性表) 153 \par
  number (数值) 153 \par
  string (字符串) 153 \par

\texttt{v:folddashes} 109

\texttt{v:foldend} 109

\texttt{v:foldstart} 109

vi 9

vi compatibility (vi 兼容性) 14, 15

Vile \par
  about (关于) 13 \par
  features (特性) 13 \par

Vim \par
  about (关于) 7, 11 \par
  advanced formatting (高级格式化) 121 \par
  autocompletion (自动补全) 84 \par
  charityware license (慈善授权) 15 \par
  color scheme, changing (修改配色方案) 21 \par
  command line buffer (命令行寄存器) 26 \par
  configuration files (配置文件) 18 \par
  download link (下载链接) 8 \par
  editor area, personalizing (个性化的编辑区) 37 \par
  extensibility (可扩展性) 141 \par
  features (特性) 12 \par
  fonts, changing (修改字体) 20 \par
  hidden markers (隐藏标记) 71 \par
  mail program (邮件程序) 210 \par
  marks, adding (添加标记) 68 \par
  matching (匹配) 22 \par
  menu, adding (添加菜单) 29-32 \par
  menu, toggling (切换菜单) 28, 29 \par
  personal highlighting (个性化高亮) 22, 23 \par
  personalizing (个性化) 17 \par
  scripting tips (脚本开发技巧) 182 \par
  script structure (脚本结构) 175 \par
  search (搜索) 63 \par
  status line (状态行) 26 \par
  syntax-color schemes (语法配色方案) 141 \par
  tabs, modifying (修改标签页) 33-37 \par
  toolbar icons, adding (添加工具栏图标) 32, 33 \par
  toolbar, toggling (切换工具栏) 28, 29 \par
  using, as Twitter client (Twitter 客户端) 212 \par
  visible markers (可见的标记) 68-70 \par

Vimballs \par
  creating (创建) 190 \par

\texttt{vimdiff} \par
  about (关于) 112 \par
  navigation (导航) 113 \par
  using, to track changes (跟踪差异) 111 \par

\texttt{vimdiff} session (会话) 112

Vim documentation (Vim 文档) 191, 193

Vim games (Vim 游戏) \par
  about (关于) 201 \par
  Game of Life (生命游戏) 202 \par
  Mines (扫雷) 204 \par
  Nibbles (贪吃蛇) 202 \par
  Rubik's cube (魔方) 203 \par
  Sokoban (推箱子) 205 \par
  Tetris (俄罗斯方块) 206 \par
  Tic-Tac-Toe (井字棋) 204 \par

VimIRC 211, 212

\texttt{vimrc} \par
  about (关于) 18, 19 \par
  cleaning, tips (清理技巧) 215 \par
  online storing (在线存储) 221 \par

\texttt{vimrc} file, cleaning tips (\texttt{vimrc} 清理技巧) \par
  comments, using (使用注释) 216 \par
  data, grouping (数据分组) 216 \par
  multiple files, using (使用多个文件) 216 \par
  Vim, using in noncompatible mode (在非兼容模式中使用 Vim) 216 \par

\texttt{vimrc} setup system (\texttt{vimrc} 设置系统) \par

Vim script (Vim 脚本) \par
  debugging (调试) 186-188 \par
  distributing (发布) 189 \par
  external interpreters, using (外部解释器) 194 \par
  installing (安装) 148, 149 \par
  scripting tips (脚本开发技巧) 182 \par
  structure (结构) 175 \par
  types (类型) 148 \par
  uninstalling (卸载) 150 \par
  using (使用方法) 147 \par

Vim scripting, in Perl (使用 Perl 开发 Vim 脚本) 195, 196

Vim scripting, in Python (使用 Python 开发 Vim 脚本) 196, 197

Vim scripting, in Ruby (使用 Ruby 开发脚本) 198, 199

Vimi script structure (Vim 脚本结构) \par
  about (关于) 175 \par
  functions (函数) 178, 179 \par
  key mappings (按键映射) 178, 179 \par
  script configuration (脚本配置) 177 \par
  script header (脚本头部信息) 176 \par
  script-loaded check (脚本加载检查) 176 \par

visible markers (可视化标记) \par
  about (关于) 68-70 \par
  \texttt{sign}, using (使用 \texttt{sign}) 68-70 \par

visual cursor, editor area (可视化光标) \par
  adding (添加) 37 \par

\section*{W}

\texttt{while} loop (\texttt{while} 循环) 166, 167

\section*{X}

xvile 13

\end{multicols}
