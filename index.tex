% vim: ts=4 sts=4 sw=4 tw=80
\chapter{索引}
\label{chap:index}

\marginpar{223}
索引中的页码指的是英文原版的页码, 与本书页边标注的页码一致.

\begin{multicols}{2}

\section*{符号}

\texttt{:diffsplit} 112

\texttt{:diffthis} 112

\texttt{:pop} 82

\texttt{:ptselect} 82

\texttt{:tag} 82

\texttt{:tnext} 82

\texttt{:tprev} 82

\texttt{:tselect} 82

\texttt{:vert diffsplit} 112

\texttt{<cWROD>} 61

\texttt{<Leader>} 178

\texttt{<Plug>} 178

\texttt{<silent>} 31

\texttt{<unique>} 178

\section*{A}

\texttt{AbbrAsk} 49

\texttt{amenu} 33

\texttt{argnum} 171

Aspell 40

autocompletion
  about (关于) 84
  all-in-one completion (多合一补全) 89, 90
  dictionary completion (字典补全) 85,86
  known word completion (已知单词补全) 84,85
  omnicompletion 86-88

\texttt{autoindent} 130

\texttt{autoproto.vim} 84

\section*{B}

balloons 43

Berkeley Par, external formatting tools (外部格式化工具) 137

black hole register (黑洞寄存器) 102

\texttt{buffers} 32

\section*{C}

\texttt{calcValue} 81

charityware license (慈善授权) 15

cindent
  about (关于) 131
  setup options (配置选项) 131

code block formatting (代码块格式化)
  commands (命令) 133

code, formatting (代码格式化)
  about (关于) 129
  \texttt{autoindent} 130
  cindent 131
  code-block, formatting (代码块格式化) 132-134
  \texttt{indentexpr} 132
  pasted code, auto formatting (自动格式化粘贴的代码) 135
  settings (设置) 130
  \texttt{smartindent} 130

color scheme (配色方案) 147

color scheme, Vim (Vim 的配色方案)
  changing (修改) 21

configuration files (配置文件)
  \texttt{exrc} 19
  \texttt{gvimrc} 19
  types (配置文件的类型) 18
  \texttt{vimrc} 18,19

context-aware navigation (基于上下文的导航)
  about (关于) 54
  code file, moving around within (在代码文件内导航) 54, 55
  code file, moving in (在代码文件内导航) 56-58

ctags command-line program (ctags 命令行程序) 80

\texttt{ctags.vim} 83

cURL 213

\texttt{cursorline} 38

\section*{D}

debugger commands (调试器命令)
  about (关于) 187
  \texttt{cont} 187
  \texttt{finish} 187
  \texttt{interrupt} 187
  \texttt{next} 187
  \texttt{quit} 187
  \texttt{step} 187

debugging (调试)
  Vim scripts (Vim 脚本) 186-188

drop registers (投递寄存器) 102

\section*{E}

editor area, Vim (Vim 的编辑器区域)
  abbreviations, using (使用缩写) 46-48
  key bindings, modifying (修改按键绑定) 49, 51
  line numbers, adding (添加行号) 39
  personalizing (个性化) 37
  spell check (拼写检查) 40-42
  tooltip, adding (工具提示) 43-46
  editor area, Vim visual cursor, adding (添加可视化光标) 37, 38

Elvis
  about (关于) 10
  features (特性) 10

Emacs editor (Emacs 编辑器) 13

expression register (表达式寄存器) 103

\texttt{exrc} 19

external formatting tools (外部格式化工具)
  about (关于) 136
  Berkeley Par 137
  Indent 136
  Tidy 138
  using (使用方法) 136

external interpreter (外部解释器)
  using in Vim scripting (在 Vim 脚本中使用) 194

\section*{F}

file explorer (文件浏览器) 208

file navigation (文件导航)
  about (关于) 54
  context-aware navigation (基于上下文的导航) 54
  long line, navigating (长行导航) 59

File-Type plugins group (文件类型插件组) 148

fold (折叠)
  about (关于) 107
  \texttt{diff}, using to track changes (使用 \texttt{diff} 跟踪差异) 114
  simple text file outlining (简单的文本文件提纲) 110, 111
  types (类型) 107
  using (使用方法) 107-109
  \texttt{vimdiff}, using to track changes (使用 \texttt{vimdiff} 跟踪差异) 111

\texttt{foldclose()} 45

fonts, Vim (Vim 字体)
  changing (修改) 20

\texttt{for} loop (\texttt{for} 循环) 164, 165

\texttt{formatexpr} 123

formatting, Vim (Vim 格式化)
  code, formatting (代码格式化) 129
  external formatting tools, using (使用外部格式化工具) 136
  text, formatting (文本格式化) 121

functions (函数)
  creating (创建) 168-170
  variable argument list (可变参数列表) 170-172

\section*{G}

Game of Life (生命游戏) 202

\texttt{get} 163

Global plugins group (全局插件组) 148

\texttt{guitablabel} 属性 36

\texttt{gvimrc} 19

\section*{H}

hacker (黑客) 15

hacking 15

\texttt{hasmapto()} 178

\texttt{helpgrep} 67

hidden markers (隐藏标记)
  about (关于) 71
  marks, using (使用标记) 71

\section*{I}

IDE (集成开发环境) 208

\texttt{indentexpr} 132

Indent, external formatting tools (外部格式化工具) 136

integrated compiler (集成的编译器) 207

integrated debugger (集成的调试器) 207

interpreter (解释器) 15
Ispell 40

\section*{J}

\texttt{join} 163

\section*{K}

key bindings, editor area (按键绑定)
  modifying (修改) 49, 51

\texttt{keys()} 165

\texttt{keyvar} 165

\section*{L}

line numbers, editor area (行号)
  adding (添加) 39

\texttt{loaded\_myscript} 176

\texttt{LoadTemplate} 77

\texttt{lookupfile.vim} 83

loops (循环)
  about (关于) 164
  \texttt{for} loop (\texttt{for} 循环) 164
  types (类型) 164
  \texttt{while} loop (\texttt{while} 循环) 164

\section*{M}

marcro recording (宏录制)
  about (关于) 90
  using (使用方法) 90-92

\texttt{map} 30, 163

\texttt{marks} (书签) 71

matching features, Vim (Vim 的匹配特性) 22

\texttt{menu} 30

MicroEmacs code (MicroEmacs 的代码) 13

Mines (扫雷) 204

MS Visual Studio\textregistered 207

Mutt 210

\texttt{mydict} dictionary (字典 \texttt{mydict}) 165

\texttt{MyIndenter()} 132

\section*{N}

named registers (命名寄存器) 101

navigation, in multiple buffers (在多个缓冲区中导航) 61

navigation, in Vim help (在 Vim 帮助系统中导航) 60

Nibbles (贪吃蛇) 202

numbered registers (编号的寄存器) 100

nvi
  about (关于) 10
  features (特性) 11

\section*{O}

omnicompletion 86-88

\section*{P}

Perl 195

personal hightlighting, Vim (Vim 的修改化高亮)
  about (关于) 22, 23
  color character, marking (彩色字符标记) 24
  errors, preventing (抑制错误) 26
  tabs not used for indentation, marking (标记不是用作缩进的制表符) 25

\texttt{PrintSum} 169

project browser (项目浏览器) 208

Python 196

\section*{R}

\texttt{range()} 164

read-only registers (只读寄存器) 101

registers (寄存器)
  about (关于) 98
  drop registers (投递寄存器) 102
  expression registers (表达式寄存器) 103
  named registers (命名寄存器) 101
  numbered registers (带编号的寄存器) 100
  read-only registers (只读寄存器) 101
  search pattern register (搜索模式寄存器) 102
  selection register (选择寄存器) 102
  small delete register (小删除寄存器) 100
  unnamed register (匿名寄存器) 100
  using (使用方法) 99

remote files (远程文件)
  editing (编辑) 117, 118
  working in (处理远程文件) 115-117

Rubik's cube (魔方) 203

Ruby 198

\section*{S}

script (脚本) 15

script development (脚本开发)
  about (关于) 150, 151
  script writing basic (脚本开发基础) 151

scripting tips, Vim (Vim 脚本开发技巧)
  Gvim, using (使用 Gvim) 182
  longer lines, printing (打印长行) 185
  multiple operating system, working with (处理多个操作系统) 183
  versions, of Vim (Vim 的版本) 183

script writing basics (脚本开发基础)
  about (关于) 151
  conditions (条件) 157, 158
  dictionaries, working with (处理字典) 159-163
  functions, creating (创建函数) 168
  lists, working with (处理线性表) 159-163
  loops (循环) 164
  types (类型) 152
  variables (变量) 153-157

\texttt{scrollbind} 113

search pattern register (搜索模式寄存器) 102

search, Vim (Vim 搜索)
  examples (例子) 64, 65
  help system, searching (在帮助系统中搜索) 67
  searching, in current file (在当前文件中搜索) 64
  searching, in multiple files (在多个文件中搜索) 65, 66

selection registers (选择寄存器) 102

\texttt{sessionoptions}
  \texttt{blank} 96
  \texttt{buffers} 96
  \texttt{curdir} 96
  \texttt{folds} 96
  \texttt{globals} 96
  \texttt{help} 96
  \texttt{localoptions} 96
  \texttt{options} 96
  \texttt{resize} 96
  \texttt{sesdir} 96
  \texttt{slash} 96
  \texttt{tabpages} 96
  \texttt{unix} 96
  \texttt{winpos} 96
  \texttt{winsize} 96

sessions (会话)
  \texttt{sessionoptions} 96
  simple session usage (会话的简单用法) 93-95
  using (使用方法) 93
  using, as project manager (项目管理程序) 97, 98

setup options, cindent (cindent 的设置选项)
  \texttt{cinkeys} 131
  \texttt{cinoptions} 131
  \texttt{cinwords} 131

\texttt{ShortTabLine()} 35

\texttt{sign} 68

Single Unix Specification (SUS, 单一 Unix 规范) 9

small delete register (小删除寄存器) 100

\texttt{smartindent} 130

snipMate plugin (snipMate 插件) 79

snipMate system (snipMate 系统) 79

Sokoban (推箱子) 205

\texttt{sort()} 165

\texttt{spelllang} 40

\texttt{spellsuggest()} 45

\texttt{split} 163

status line, Vim (Vim 的状态行) 26, 28

STEVIE 9

\texttt{suffixadd} 62

\texttt{sum} 170

syntax coloring (语法高亮) 142, 143, 147

syntax-color scheme (语法高亮主题) 141

syntax regions (语法区) 143-146

\section*{T}

tabs, Vim (Vim 的标签页)
  modifying (修改) 33-37

tag browser (Tag 浏览器) 208

tag list generators (Tag list 生成程序)
  about (关于) 80
  Ctags 80
  Hdrtags 80
  Jtags 80
  Ptags 80
  Vtags 80

tag lists
  about (关于) 80
  taglist navigation (taglist 导航) 83
  uses 83
  using (使用方法) 80-82

\texttt{taglist.vim} 83

templates (模版)
  abbreviations, using (使用缩写) 76, 77
  about (关于) 74
  snippets, with snipMate script (通过 snipMate 使用代码片断) 78, 79
  template files, using (使用模版文件) 74, 75

Tetris (俄罗斯方块) 206

text, formatting (文本格式化)
  about (关于) 121
  headlines, marking (标题行标记) 125, 126
  lists, creating (创建线性表) 127-129
  text, aligning (文本对齐) 124
  text, putting into paragraph (文本分段) 122, 123

The Mail Suite (TMS, 邮件套装) 210

Tic-Tac-Toe (井字棋) 204

Tidy, external formatting tools (外部格式化工具 Tidy) 138

TwitVim 212

\section*{U}

undo branching (撤消分支)
  about (关于) 98
  using (使用方法) 103-106

unnamed register (匿名寄存器) 100

\section*{V}

variables (变量)
  about (关于) 153
  dictionary (字典) 153
  \texttt{funcref} 153
  list (线性表) 153
  number (数值) 153
  string (字符串) 153

\texttt{v:folddashes} 109

\texttt{v:foldend} 109

\texttt{v:foldstart} 109

vi 9

vi compatibility (vi 兼容性) 14, 15

Vile
  about (关于) 13
  features (特性) 13

Vim
  about (关于) 7, 11
  advanced formatting (高级格式化) 121
  autocompletion (自动补全) 84
  charityware license (慈善授权) 15
  color scheme, changing (修改配色方案) 21
  command line buffer (命令行寄存器) 26
  configuration files (配置文件) 18
  download link (下载链接) 8
  editor area, personalizing (个性化的编辑区) 37
  extensibility (可扩展性) 141
  features (特性) 12
  fonts, changing (修改字体) 20
  hidden markers (隐藏标记) 71
  mail program (邮件程序) 210
  marks, adding (添加标记) 68
  matching (匹配) 22
  menu, adding (添加菜单) 29-32
  menu, toggling (切换菜单) 28, 29
  personal highlighting (个性化高亮) 22, 23
  personalizing (个性化) 17
  scripting tips (脚本开发技巧) 182
  script structure (脚本结构) 175
  search (搜索) 63
  status line (状态行) 26
  syntax-color schemes (语法配色方案) 141
  tabs, modifying (修改标签页) 33-37
  toolbar icons, adding (添加工具栏图标) 32, 33
  toolbar, toggling (切换工具栏) 28, 29
  using, as Twitter client (Twitter 客户端) 212
  visible markers (可见的标记) 68-70

Vimballs
  creating (创建) 190

\texttt{vimdiff}
  about (关于) 112
  navigation (导航) 113
  using, to track changes (跟踪差异) 111

\texttt{vimdiff} session (会话) 112

Vim documentation (Vim 文档) 191, 193

Vim games (Vim 游戏)
  about (关于) 201
  Game of Life (生命游戏) 202
  Mines (扫雷) 204
  Nibbles (贪吃蛇) 202
  Rubik's cube (魔方) 203
  Sokoban (推箱子) 205
  Tetris (俄罗斯方块) 206
  Tic-Tac-Toe (井字棋) 204

VimIRC 211, 212

\texttt{vimrc}
  about (关于) 18, 19
  cleaning, tips (清理技巧) 215
  online storing (在线存储) 221

\texttt{vimrc} file, cleaning tips (\texttt{vimrc} 清理技巧)
  comments, using (使用注释) 216
  data, grouping (数据分组) 216
  multiple files, using (使用多个文件) 216
  Vim, using in noncompatible mode (在非兼容模式中使用 Vim) 216

\texttt{vimrc} setup system (\texttt{vimrc} 设置系统)

Vim script (Vim 脚本)
  debugging (调试) 186-188
  distributing (发布) 189
  external interpreters, using (外部解释器) 194
  installing (安装) 148, 149
  scripting tips (脚本开发技巧) 182
  structure (结构) 175
  types (类型) 148
  uninstalling (卸载) 150
  using (使用方法) 147

Vim scripting, in Perl (使用 Perl 开发 Vim 脚本) 195, 196

Vim scripting, in Python (使用 Python 开发 Vim 脚本) 196, 197

Vim scripting, in Ruby (使用 Ruby 开发脚本) 198, 199

Vimi script structure (Vim 脚本结构)
  about (关于) 175
  functions (函数) 178, 179
  key mappings (按键映射) 178, 179
  script configuration (脚本配置) 177
  script header (脚本头部信息) 176
  script-loaded check (脚本加载检查) 176

visible markers (可视化标记)
  about (关于) 68-70
  \texttt{sign}, using (使用 \texttt{sign}) 68-70

visual cursor, editor area (可视化光标)
  adding (添加) 37

\section*{W}

\texttt{while} loop (\texttt{while} 循环) 166, 167

\section*{X}

xvile 13

\end{multicols}
