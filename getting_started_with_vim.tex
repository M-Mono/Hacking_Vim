% vim: ts=4 sts=4 sw=4 et tw=75
\chapter{开始}
\label{chap:getting_started_with_vim}

\marginpar{7}
\newterm{Vim} (\newterm{Vi IMproved}) 编辑器最早由 Bram Moolenaar 于
1991 年 11 月发布, 当时只是作为 Unix vi 编辑器的 Amiga 平台克隆版.

一年后, Unix 平台的 Vim 发布, 之后, 它迅速成为了 vi 的替代版本.

由于宽松的授权和丰富的功能, 在开源社区的帮助下, Vim 逐渐流行起来.
越来越多的 Linux 发行版开始用 Vim 替换掉 vi. 虽然许多用户认为他们使用
的是 vi (如果他们是通过执行命令 \vi 来打开编辑的话), 可实际上
打开的是 Vim (命令 \vi 已经被 \vim 的链接替换掉, 所以经常会有人误以为
vi 和 Vim 是同一个程序).

在九十年代后期, vi 在编辑器之战中所输掉的劣势, 重新又被 Vim 给赢了回来,
编辑器之战指的是 vi 和 Emacs 之间的斗争. Bram 为 Vim 扩充了许多新特性, 
而这些特性原本被  Emacs 党利用, 作为论证 Vim/vi 不如 Emacs 的论据, 即
使如此, Bram 仍然没有忘记当初人们开发 vi 的初衷.

如今, Vim 已经是一个功能丰富, 定制性强, 受人欢迎的编辑器. 它支持超过 200 
种语言的语法高亮, 自动补全, 折叠, 撤消/重做, 多重缓冲区/窗口/标签, 以及
其他特性.

本章主要介绍
\begin{itemize}
    \item 如何获取与安装 Vim 编辑器
    \item Vim 编辑器家族
    \item Vim 的发布许可证 
    \item 本书使用的公共术语
\end{itemize}
现在, 让我们正式开始.

\marginpar{8}
\section{获取 Vim}
\label{sec:getting_vim}

读者也许对 Vim 有了一定的了解, 而且也使用了一段时间, 然而, 如果你还没有
使用过 Vim, 那么最好趁现在这个时候, 在自己的系统中安装 Vim.

你可以从网站 \url{http://www.vim.org} 下载到 Vim 的最新版.
\begin{warning}
本书主要讨论 Vim 7.2, 如果你用的 版本比较老, 请不要担心, 你可以随时
更新到最新版.
\end{warning}
如果你所用的操作系统是 Microsoft Windows, 只需要双击运行下载的 \texttt{.exe}
文件, 就可以开始安装过程. 安装完毕后, 在 ``开始'' 菜单中就会出现一个指
向 gVim 的快捷键.

如果是 Linux, 那么安装方式取决于你所使用的 Linux 发行版. 如今, 在大多数
发行版中已经预先安装了 Vim, 如果没有, 具体的安装方法请参考发行版
的软件包管理器 (例如, debian 的软件包管理器是 Aptitude, Mandriva 是
urpmi, Ubuntu 的是 Synaptics). 如果系统中没有软件包管理器, 你可以从
上面所提供的网站中下载 Vim 的源代码, 手工编译安装, 具体的编译安装方法
可以看一下源码包中的 \file{readme} 文件.

\section{vi, Vim, 及其朋友}
\label{sec:vi_vim_and_friends}
vi 最早由 Bill Joy 于 1976 年发布, Vim 只是 vi 众多的衍生版之一. 其中一些
衍生版的特性和 vi 非常接近, 而另一些则新增了许多新特性, Vim 就属于后者.
接下来, 我们将会介绍一些比较著名 的 vi 衍生版, 并简要描述每个衍生版的
特点.
\marginpar{9}
\subsection{vi}
\label{subsec:vi}
vi 是 Vim 家族的原始祖先, 由 Bill Joy 在 1976 年开发, 系统平台是 
\newterm{BSD} (\newterm{Berkeley Software Distribution}) 的一个早期版本.
vi 是当时最流行的编辑器 \newterm{ex} 的扩展版本. 而 ex 则是 Unix 编辑器 
\newterm{ed} 的扩展版本. ``vi'' 的意思是 \newterm{visual in ex}, 正如这个
名字所表示的那样, vi 仅仅是一个命令, 命令的作用是以可视化模式启动 ex 
编辑器.

vi 是最早引入\newterm{模式} (\newterm{modality}) 概念的编辑器之一.
模式指的是在处理不同的任务时, 编辑器可以处于不同的工作模式 --- 有的模式
用来编辑文本, 有的模式用来选择文本, 还有一些模式用于执行命令.

模式是 vi 的主要特性之一, 这个特性使得热爱 vi 的人更加热爱它, 但也会让
讨厌 vi 的人更加讨厌.

自从第一个版本发布 之后, vi 就没有发生过比较大的变化, 但这并不妨碍它成
为 Unix 社区最流行的编辑器之一, 这主要是因为 \newterm{SUS} (\newterm{Single
Unix Specification}) 把 vi 列为 Unix 系统的必备软件之一 --- 只有符合 
SUS 的系统才能称之为 Unix 系统.

\subsection{STEVIE}
\label{subsec:stevie}
1987 年, Tim Thompson 获得了他的第一台 \newterm{Atari ST}
(\newterm{Sixteen/Thirty-two}), 但这个平台还没有一款优秀的编辑器可供
使用, 于是, 他决定把 vi 移值到这个平台中. 1987 年 6 月, 他发布了一款
编辑器, 所使用的授权类似于后来的开源协议. 他在新闻组中发布了这款编辑器,
并取名为 \newterm{STEVIE} --- 意为 \newterm{ST Editor for VI Enthusiasts}.

这款编辑器非常简单, 只支持 vi 的一小部分功能, 但是它提供了一个对 vi 用
户来说非常熟悉的环境, 这可以帮助他们在 ST 平台上继续高效地工作.

发布之后, Tim Thompson 停止了 STEVIE 的开发工作, 但是 Tony Andrews 马上
接手了过来, 并且在一年内, 就把它移植到了 Unix 和 OS/2 系统中. 在这过程中,
越来越多的特性被加了进来, 但是到了 1990 年, 开发工作又停止了.

虽然 STEVIE 只生存了几年, 但是 Tim 和 Tony 把源代码放到了新闻组上,
任何人都可以免费地浏览和下载, 正因为如此, 后来的许多 vi 衍生版都或多或
少从这些代码中得到启发, 或以它们为基础再加以开发.
\marginpar{10}

\section{Elvis}
\label{sec:elvis}
STEVIE 是比较流行的编辑器之一, 但是它还有许多 bug, 限制也比较多. 当时 
还在使用 Minix 的 Steve Kirkendall 注意到了 STEVIE 的一个缺点: 在编辑文件
时, 它会把整个文件读取到内存中, 对 Minix 来说这并不是一个很明智的做法.
于是 Steve 决定修改 STEVIE, 让编辑器把一个文件当做缓冲区使用, 而不是
在内存中编辑, 这产生了 \newterm{Elvis} v1.0.

虽然和 vi 比起来, Elvis 已经得到了很大的改善, 但是它仍然受到同样的限制
--- 行的最大长度, 以及单文件缓冲区.

为了完全摆脱这些限制, Steve Kirkendall 决定重写 Elvis, 这产生了 Elvis
v2, 当前可用的版本是 v2.2.

在 Elvis 的第 2 版中, Steve 添加了许多 vi 原来没有的特性, 其中比较重要
的有:
\begin{itemize}
\item 语法高亮
\item 多窗口支持
\item 网络支持 (HTTP 和 FTP)
\item 简单的 GUI 前端
\end{itemize}
Elvis 现在已经停止了开发, 但仍然被广泛地使用, Elvis 支持的平台包括
Unix, MS Windows (控制台程序, 或带有 GUI 的 WinElvis), 以及 OS/2.
\begin{warning}
关于 Elvis 的最新版请访问
\url{http://elvis.the-little-red-haired-girl.org/}.
\end{warning}

\section{nvi}
\label{sec:nvi}
\newterm{nvi}, 全称 \newterm{new vi}, 是 AT\&T 与 加州大学伯克利分校 
\newterm{Computer Science Research Group} (\newterm{CSRG}) 授权争论的
结果. vi 使用了 ed 的源代码, 而 ed 使用了 AT\&T System V Unix 授权,
所以 CSRG 无法使用 BSD 授权来发布 vi, 于是他们决定发布 vi 的替代版本.
\marginpar{11}

新 vi 的开发人员是 Keith Bostic. vi 克隆了已经免费的 Elvis, 但是 
Keith 希望新的编辑器和原始的 vi 尽量保持接近, 于是他采用了 Elvis 的代码,
并开发出一个和 vi 完全兼容的衍生版 --- nvi. 在原始  vi 的功能集
中, 只能 \newterm{打开模式} (\newterm{Open Mode}) 和 \newterm{lisp 编辑}
(\newterm{lisp edit}) 被剔除出去.

随着 4.4BSD 的发布, nvi 完全替代了 vi 编辑器, 而且再次使用了完全免费的
授权来发布该软件.

如今, 大多数基于 BSD 的操作系统都使用了 nvi 作为它们的默认 vi 编辑器,
包括 NetBSD, FreeBSD, 还有 OpenBSD, 并且功能也越来越丰富.

和原始的 vi 相比, nvi 包含的新特性主要有:
\begin{itemize}
    \item 多个编辑缓冲区
    \item 不限次数的撤消
    \item 扩展的正则表达式
    \item 支持 CScope
    \item 简单的脚本支持, 脚本语言可以是 Perl 或 Tcl/Tk
\end{itemize}

Keith Bostic 现在依然在维护 nvi 的源代码, 但是开发量已经少了很多.
\begin{warning}
    可以到 \url{http://www.bostic.com/vi/} 获取最新版的 nvi.
\end{warning}

\section{Vim}
\label{sec:vim}
\newterm{Vim} 编辑器是 vi 家族的天之骄子. 自从 Bram Moolenaar 在 1991 年
发布了 Vim 的第 1 版之后, 这个编辑器已经成长为世界上功能最丰富的编辑器
之一.
\marginpar{12}
