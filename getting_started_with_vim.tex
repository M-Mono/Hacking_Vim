% vim: ts=4 sts=4 sw=4 et tw=75
\chapter{开始}
\label{chap:getting_started_with_vim}

\marginpar{7}
Vim (Vi IMproved) 编辑器最早由 Bram Moolenaar 于 1991 年 11 月发布,
当时只是作为 Unix vi 编辑器的 Amiga 平台克隆版.

一年后, Unix 平台的 Vim 发布, 之后, 它迅速成为了 vi 的替代版本.

由于宽松的授权和丰富的功能, 在开源社区的帮助下, Vim 逐渐流行起来.
越来越多的 Linux 发行版开始用 Vim 替换掉 vi. 虽然许多用户认为他们使用
的是 vi (如果他们是通过执行命令 \vi 来打开编辑的话), 可实际上
打开的是 Vim (命令 \vi 已经被 \vim 的链接替换掉, 所以经常会有人误以为
vi 和 Vim 是同一个程序).

在九十年代后期, vi 在编辑器之战中所输掉的劣势, 重新又被 Vim 给赢了回来,
编辑器之战指的是 vi 和 Emacs 之间的斗争. Bram 为 Vim 扩充了许多新特性, 
而这些特性原本被  Emacs 党利用, 作为论证 vim/vi 不如 Emacs 的论据, 即
使如此, Bram 仍然没有忘记当初人们开发 vi 的初衷.

如今, Vim 已经是一个功能丰富, 定制性强, 受人欢迎的编辑器. 它支持超过 200 
种语言的语法高亮, 自动补全, 折叠, 撤消/重做, 多重缓冲区/窗口/标签, 以及
其他特性.

本章主要介绍
\begin{itemize}
    \item 如何获取与安装 Vim 编辑器
    \item Vim 编辑器家族
    \item Vim 的发布许可证 
    \item 本书使用的公共术语
\end{itemize}
现在, 让我们正式开始.

\marginpar{8}
\section{获取 Vim}
\label{sec:getting_vim}

读者也许对 Vim 有了一定的了解, 而且也使用了一段时间, 然而, 如果你还没有
使用过 Vim, 那么最好趁现在这个时候, 在自己的系统中安装 Vim.

你可以从网站
\begin{gencode}
http://www.vim.org
\end{gencode}
下载到 Vim 的最新版.
\begin{waring}
本书主要讨论 Vim 7.2, 如果你用的 版本比较老, 请不要担心, 你可以随时
更新到最新版.
\end{waring}
如果你所用的操作系统是 Microsoft Windows, 只需要双击运行下载的 \texttt{.exe}
文件, 就可以开始安装过程. 安装完毕后, 在 ``开始'' 菜单中就会出现一个指
向 gVim 的快捷键.

如果是 Linux, 那么安装方式取决于你所使用的 Linux 发行版. 如今, 在大多数
发行版中已经预先安装了 Vim, 如果没有, 具体的安装方法请参考发行版
的软件包管理器 (例如, debian 的软件包管理器是 Aptitude, Mandriva 是
urpmi, Ubuntu 的是 Synaptics). 如果系统中没有软件包管理器, 你可以从
上面所提供的网站中下载 Vim 的源代码, 手工编译安装, 具体的编译安装方法
可以看一下源码包中的 \file{readme} 文件.

\section{vi, Vim, 及其朋友}
\label{sec:vi_vim_and_friends}
vi 最早由 Bill Joy 于 1976 年发布, Vim 只是 vi 众多的衍生版之一. 其中一些
衍生版的特性和 vi 非常接近, 而另一些则新增了许多新特性, Vim 就属于后者.
接下来, 我们将会介绍一些比较著名 的 vi 克隆版, 并简要描述每个克隆版的
特点.
\marginpar{9}
