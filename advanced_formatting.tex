% vim: ts=4 sts=4 sw=4 tw=80
\chapter{高级格式化}
\label{chap:advanced_formatting}
\marginpar{121}

很多时候, 最简单的工作就是把文本或代码修改成更易阅读的形式. 在这一章, 我们将会
介绍一些简单的方法, 用于对文本进行格式化 --- 无论是普通文本还是代码.

这一章主要分为三个部分:
\begin{itemize}
    \item 文本格式化
    \item 代码格式化
    \item 使用外部格式化工具
\end{itemize}

学完这一章后, 用户应该可以清楚地知道当要对文本进行格式化时, 什么情况下应该用
Vim, 什么情况下不应该用 Vim.

\section{格式化文本}
\label{sec:formatting_text}

在编辑普通文本时, 虽然大多数人更喜欢使用图形化字处理软件, 比如 Microsoft Word
或 OpenOffice, 但是许多纯文本编辑器, 比如 Vim, 也可以把事情做得很好. 在下面的
一节里, 我们将会介绍如何利用 Vim 强大的功能来格式化普通文本.
\marginpar{122}

\subsection{文本分段}
\label{subsec:putting_text_into_paragraphs}

这一节所介绍的知识可能是整本书中最简单的, 但是对于格式化普通文本来说, 却可能是
最有用的. 假设用户正在编写一段文本, 并且在编写的过程中丝毫没有注意到行的变化与
文本的格式. 最终, 用户可能会写出一行非常长的文本, 此时他才注意到应该对文本重新
进行格式化, 摆在他面前的有两个选择:
\begin{itemize}
    \item 遍历文本, 并在适当的地方断行
    \item 用某个命令对整段文本进行格式化
\end{itemize}
很显然, 后者是最好的选择, 并且格式化后的结果也可以和其他结果保持一致. 所使用的
命令是:
\begin{vimcode}
gqap
\end{vimcode}
上面的命令其时是两个命令的组合:
\begin{itemize}
    \item \texttt{gq}: 对所有的,
\end{itemize}
