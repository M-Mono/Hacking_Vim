% vim: ts=4 sts=4 sw=4 tw=80
\chapter{高级格式化}
\label{chap:advanced_formatting}
\marginpar{121}

很多时候, 最简单的工作就是把文本或代码修改成更易阅读的形式. 在这一章, 我们将会
介绍一些简单的方法, 用于对文本进行格式化 --- 无论是普通文本还是代码.

这一章主要分为三个部分:
\begin{itemize}
    \item 文本格式化
    \item 代码格式化
    \item 使用外部格式化工具
\end{itemize}

学完这一章后, 用户应该可以清楚地知道当要对文本进行格式化时, 什么情况下应该用
Vim, 什么情况下不应该用 Vim.

\section{格式化文本}
\label{sec:formatting_text}

在编辑普通文本时, 虽然大多数人更喜欢使用图形化字处理软件, 比如 Microsoft Word
或 OpenOffice, 但是许多纯文本编辑器, 比如 Vim, 也可以把事情做得很好. 在下面的
一节里, 我们将会介绍如何利用 Vim 强大的功能来格式化普通文本.
\marginpar{122}

\subsection{文本分段}
\label{subsec:putting_text_into_paragraphs}

这一节所介绍的知识可能是整本书中最简单的, 但是对于格式化普通文本来说, 却可能是
最有用的. 假设用户正在编写一段文本, 并且在编写的过程中丝毫没有注意到行的变化与
文本的格式. 最终, 用户可能会写出一行非常长的文本, 此时他才注意到应该对文本重新
进行格式化, 摆在他面前的有两个选择:
\begin{itemize}
    \item 遍历文本, 并在适当的地方断行
    \item 用某个命令对整段文本进行格式化
\end{itemize}
很显然, 后者是最好的选择, 并且格式化后的结果也可以和其他结果保持一致. 所使用的
命令是:
\begin{vimcode}
gqap
\end{vimcode}
上面的命令其时是两个命令的组合:
\begin{itemize}
    \item \texttt{gq}: Format everything the next movement moves
        over\footnote{TODO}
    \item \texttt{ap}: ``A paragraph'' moves over the current
        paragraph\footnote{TODO}
\end{itemize}

换句话说, 由 \texttt{gq} 与 \texttt{ap} 组合而成的命令告诉 Vim 去遍历当前段落
并格式化. 由两个空白行包围起来的部分定义为一个段落, 因此, 为了开始一个新的段
落, 只需要添加一个空行即可.

Vim 所做的格式化实际上是让行作出更漂亮的断行, 使得每一行的长度都不会超过某个
特定的大小 (Vim 自动地在适当的两个单词间作出断行).

格式化后文本的宽度由选项 \texttt{textwidth} 定义, 如果用户希望每行最大的宽度不
超过 80 个字符, 那就在 \texttt{vimrc} 中添加:
\begin{vimcode}
:set textwidth=80
\end{vimcode}
如果 \texttt{textwidth} 的值被设置为 0, 那么 Vim 就把它设置成窗口的宽度 ---
但是永远不会多于 \texttt{textwidth} 所设置的字符个数.\footnote{If the option is
    set to 0, the Vim sets it to the width of the window --- but never more than
    the number of characters defined in the the \texttt{textwidth} setting.}
\marginpar{123}

\begin{warning}
    通过设置选项 \texttt{formatoptions} 可以控制 Vim 如何格式化段落. 更多的信
    息可以参考 \texttt{:help 'formatoptions'} 与 \texttt{:help 'fo-table'}.
\end{warning}

\texttt{gq} 可以和任意的移动命令配合使用, 并且在格式化之后, 光标将会停留在移动
命令结束的地方 (典型的情况是停留在当前特定区域的最后一行). 如果用户希望在格式
化后, 光标仍旧处于格式化开始前的位置, 那就把 \texttt{gq} 改成 \texttt{gw}. 如
果用户的光标原来是在段落中第一行的开始, 此时执行 \texttt{gwap}, 命令结束后, 光
标仍然会停留在段落中第一行的开始.

用户可以在命令的前面加上数字, 使得它重复执行, 例如, \texttt{5gqap} 将会对当前
与下面的四个段落进行格式化. 如果想要对文件内的所有段落进行格式化, 就执行
\texttt{1gqG}.

前面的介绍的格式化命令不仅对普通文本有效, 对其他任意类型的内容同样有效, 而且用
户可以决定使用哪种格式化函数.

用户可以为指定的文件类型设置任意的格式化函数, 方法是设置 Vim 的选项
\texttt{formatexpr}. 例如, 如果用户想要对 C 源代码进行格式化, 只需要在
\texttt{vimrc} 中添加:
\begin{vimcode}
:set formatexpr=c#Formatter()
\end{vimcode}
这行命令告诉 Vim, 在打开一个 C 源代码文件时, 使用函数 \texttt{Formatter()},
这个函数定义在 Vim 为 C 文件自动加载的文件中.

\begin{warning}
    自动回载的文件可以在 \texttt{VIMHOME} 的子目录 \texttt{autoload} 中找到.
    文件根据所服务的文件类型来命令, 以后缀 \texttt{.vim} 结束. 例如, 为 C 文
    件自动加载的文件是 \texttt{VIMHOME/autoload/c.vim}.
\end{warning}

一个格式化函数含有三个变量, 利用这三个变量可以找到待格式化的文本.
\begin{itemize}
    \item \texttt{v:num}: 待格式化的第一行的行号
    \item \texttt{v:count}: 待格式化的行的数量
    \item \texttt{v:char}: 这个变量包含了将被插入的字符, 可以为空
\end{itemize}

格式化函数的一个简单示例是:
\marginpar{124}
\begin{vimcode}
function! MyFormatter()
   let first = v:num
   let last = v:num + v:count
    while(first<=last)
       call setline(first, '> '. getline(first))
       let first = first+1
    endwhile
endfunction
\end{vimcode}
这个格式化文本接收所有的行, 在每一行的开始添加 \texttt{>}, 就像 E-mail 中的引
用.

上面展示的格式化函数非常简单, 如果需要更高级一点的, 那么函数的复杂度就会快速
增加, 因此公开可用的格式化函数非常有限 (这些函数是为了某些特定的目标而开发的).

\subsection{对齐文本}
\label{subsec:aligning_text}

在大多数字处理程序中, 最基本的格式化选项之一是左对齐, 右对齐, 或居中. 其中
一些程序甚至可以让文本两端对齐, 这样做可以让每一行的结束尽可能地向边缘靠近.

虽然上面提到的格式化类型对字处理程序来说非常常见, 但对于普通文本编辑器来说就
很少见了, Vim 就位列其中.

Vim 支持三种类型的对齐 --- 左对齐, 右对齐, 居中. 在我们介绍它们如何工作之前,
先简单解释一下支持这种对齐类型的文本编辑器为什么很少.

对于常见的文本编辑器来说, 它们不含有隐藏信息, 也就是说, 用户所能看到的, 就是
用户所能拥有的 --- 没有页面宽度, 没有对齐, 什么都没有.

另一方面, 在字处理程序中, 文档隐藏了相当丰富的信息, 这些信息告诉编辑器如何
根据用户的需要, 对文本进行格式化.

因为上面所说的情况对 Vim 并不适用, 所以用户得向 Vim 提供一些信息, 来帮助 Vim
进行格式化. 例如, 为了让编辑器知道对齐时所需的边缘, 用户需要设置文本的宽度.

说完了这些, 先来看一下居中排列的命令:
\begin{vimcode}
:[range]center WIDTH
\end{vimcode}
\marginpar{125}
命令中的 \texttt{range} 是用户希望居中的行的范围, \texttt{WIDTH} 是每行最多的
字符数. 典型的用法是在可视模式下选中待居中的文本 (按住 \key{Shift+v}, 然后移
动光标), 然后再输入命令. 按下 \texttt{:} 后, 用户将会发现 Vim 自动地把选中的
文本的范围记成 \texttt{'<,'>}, 这表示从选中文本的第一行到最后一行.

接下来, 用户需要输入 \texttt{center} 与文本的宽度. 如果选项 \texttt{textwidth}
已经设置妥当, 那就不需要输入 \texttt{WIDTH}.

如果选项 \texttt{textwidth} 的值为 0, 用户又没有在命令中设置 \texttt{WIDTH},
那么 Vim 就默认使用 80 个字符的宽度. 在字处理程序中, 用户很容易就可以看出文本
是否是居中的, Vim 对文本进行居中时, 只是在前面加上适当数量的空格, 这样
做意味着无论是在什么时候修改了文本, 用户都要重新对文本进行居中.

下面的命令用于左对齐:
\begin{vimcode}
:[range]left INDENT
\end{vimcode}
同样, 执行这个命令时需要提供行的范围, 如果需要的话, 再提供缩进的宽度. 这个命令
可以用来精确地设置文本的左边边缘.

最后是右对齐命令. 同样的, Vim 可能并不知道一行的宽度, 因此执行命令时还要提供
宽度信息. 命令的形式是:
\begin{vimcode}
:[range]right WIDTH
\end{vimcode}
范围内的行用空格缩进, 使得每一行的结束都是对齐的, 宽度由用户指定. 和居中一样,
无论何时修改了文本, 都要重新对文本进行右对齐.

\subsection{标记标题}
\label{subsec:marking_headlines}

使用普通的文本编辑器编写文档时, 用户可能需要创建自定义的格式或标记, 使得文本
更容易阅读.

为了提高可读性, 一个常见的操作是为那些用作标题的字符串作标记, 这些标题可以是
文本的章节.
\marginpar{126}

如果是字处理程序, 只需要把字体设置得更大更粗就可以了, 但是这样的字体设置对
Vim 却是不可能实现的, 因为 Vim 字体的大小是固定的. 因此, Vim 用户必须通过其他
的方法来标记标题.

笔者个人的选择是在标题的下面添加一行, 来表示这是一个标题行.

一个简单的例子是:
\begin{vimcode}
My Headline
===========
This is the text on the document. It could cantain one
or more lines of text.
\end{vimcode}
不同级别的标题可以用不同类型的标记来表示:
\begin{vimcode}
Level1
======
Level2
------
-Level3-
\end{vimcode}

为了更方便地添加下划线, 在 Vim 中可以用宏来实现, 使用这种方法不用担心下划线添
加得太多或太少.

前面两个级别的标题宏可以实现成:
\begin{vimcode}
yypVr=o
\end{vimcode}
宏的各个部分的具体涵义是:
\begin{itemize}
    \item \texttt{yy}: 复制当前行
    \item \texttt{p}: 粘贴
    \item \texttt{V}: 选中一整行
    \item \texttt{r}: 用后面的字符 (在这里是 \texttt{=}) 替换掉选中的字符
    \item \texttt{o}: 在光标的下面添加一个空行, 把光标移动到这行的开始, 并切
        换到插入模式
\end{itemize}

这个宏的基本功能是获取当前行 (标题行), 并复制它. 然后把复制得到的行的所有字符
替换成某个字符 (在这个例子中是 \texttt{-} 或 \texttt{=}), 最后, 插入一个新行,
并切换到插入模式.

对第三级别的标题来说, 我们必须采取其他办法, 不过所做的操作无非是在一行的开始
和结束添加一个连字符, 可以用一条替换命令来完成:
\marginpar{127}
\begin{vimcode}
:s/\.(.*\)/-\1-/
\end{vimcode}
我们把命令拆成三部分来说明:
\begin{itemize}
    \item \texttt{:s///}: 替换命令
    \item \texttt{\(.*\)}: 正则表达式, 表达式把当前行的所有字符作为输入, 并把
        它们作为搜索模式
    \item \texttt{-\\1-}: 替换模式. 替换模式告诉 Vim 在第一个被匹配的子模式前
        添加一个连字符, 在子模式的后面再加上一个连字符
\end{itemize}

记住, 这些宏可能会写得很复杂, 但我们可以很轻易地为它们绑定一个快捷键, 例如:
\begin{vimcode}
:map h1 yypVr=0
:map h2 yypVr-0
:map h3 :s/\.(.*\)/-\1-/
\end{vimcode}
现在, 用户可以在普通模式下, 通过按 \texttt{h1}, \texttt{h2}, \texttt{h3} 来添
适当的标题行. 如果用户不想在标题行的下面添加一个空行并切换到插入模式, 就把上
面的映射命令中的 \texttt{o} 删掉.

\subsection{创建列表}
\label{subsec:creating_lists}

项目列表与编号列表是文档中的常见结构. 在这一节, 我们将会介绍在 Vim 中如何方便
地创建这些列表.

让我们首先来看一个函数, 这个函数接收某个范围内被选中的行, 然后把它们转化成项
目列表. 在这个列子里, 一个简单的项目列表是:
\begin{vimcode}
* first item
* second item
* third item
\end{vimcode}
为每一行的开始添加星号的函数可以是:
\begin{vimcode}
function! BulletList()
    let lineno = line(".")
    call setline(lineno, "    * " . getline(lineno))
endfunction
\end{vimcode}
\marginpar{128}
