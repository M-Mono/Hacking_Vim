% vim: ts=4 sts=4 sw=4 tw=80
\chapter{格式化进阶}
\label{chap:advanced_formatting}
\marginpar{121}

很多时候, 最简单的工作就是把文本或代码修改成更易阅读的形式. 在这一章, 我们将会
介绍一些简单的方法, 用于对文本进行格式化 --- 无论是普通文本还是代码.

这一章主要分为三个部分:
\begin{itemize}
    \item 文本格式化
    \item 代码格式化
    \item 使用外部格式化工具
\end{itemize}

学完这一章后, 用户应该可以清楚地知道当要对文本进行格式化时, 什么情况下应该用
Vim, 什么情况下不应该用 Vim.

\section{格式化文本}
\label{sec:formatting_text}

在编辑普通文本时, 虽然大多数人更喜欢使用图形化字处理软件, 比如 Microsoft Word
或 OpenOffice, 但是许多纯文本编辑器, 比如 Vim, 也可以把事情做得很好. 在下面的
一节里, 我们将会介绍如何利用 Vim 强大的功能来格式化普通文本.
\marginpar{122}

\subsection{文本分段}
\label{subsec:putting_text_into_paragraphs}

这一节所介绍的知识可能是整本书中最简单的, 但是对于格式化普通文本来说, 却可能是
最有用的. 假设用户正在编写一段文本, 并且在编写的过程中丝毫没有注意到行的变化与
文本的格式. 最终, 用户可能会写出一行非常长的文本, 此时他才注意到应该对文本重新
进行格式化, 摆在他面前的有两个选择:
\begin{itemize}
    \item 遍历文本, 并在适当的地方断行
    \item 用某个命令对整段文本进行格式化
\end{itemize}
很显然, 后者是最好的选择, 并且格式化后的结果也可以和其他结果保持一致. 所使用的
命令是:
\begin{vimcode}
gqap
\end{vimcode}
上面的命令其时是两个命令的组合:
\begin{itemize}
    \item \texttt{gq}: 把紧接在该命令后的动作所覆盖到的文本进行格式化
    \item \texttt{ap}: 选择一个段落 (``a paragraph'')
\end{itemize}

换句话说, 由 \texttt{gq} 与 \texttt{ap} 组合而成的命令告诉 Vim 去遍历当前段落
并格式化. 由两个空白行包围起来的部分定义为一个段落, 因此, 为了开始一个新的段
落, 只需要添加一个空行即可.

Vim 所做的格式化实际上是让行作出更漂亮的断行, 使得每一行的长度都不会超过某个
特定的大小 (Vim 自动地在适当的两个单词间作出断行).

格式化后文本的宽度由选项 \texttt{textwidth} 定义, 如果用户希望每行最大的宽度不
超过 80 个字符, 那就在 \texttt{vimrc} 中添加:
\begin{vimcode}
:set textwidth=80
\end{vimcode}
如果 \texttt{textwidth} 的值被设置为 0, 那么 Vim 就把它设置成窗口的宽度 ---
但是永远不会多于 \texttt{textwidth} 所设置的字符个数.\footnote{If the option is
    set to 0, the Vim sets it to the width of the window --- but never more than
    the number of characters defined in the the \texttt{textwidth} setting.}
\marginpar{123}

\begin{warning}
    通过设置选项 \texttt{formatoptions} 可以控制 Vim 如何格式化段落. 更多的信
    息可以参考 \texttt{:help 'formatoptions'} 与 \texttt{:help 'fo-table'}.
\end{warning}

\texttt{gq} 可以和任意的移动命令配合使用, 并且在格式化之后, 光标将会停留在移动
命令结束的地方 (典型的情况是停留在当前特定区域的最后一行). 如果用户希望在格式
化后, 光标仍旧处于格式化开始前的位置, 那就把 \texttt{gq} 改成 \texttt{gw}. 如
果用户的光标原来是在段落中第一行的开始, 此时执行 \texttt{gwap}, 命令结束后, 光
标仍然会停留在段落中第一行的开始.

用户可以在命令的前面加上数字, 使得它重复执行, 例如, \texttt{5gqap} 将会对当前
与下面的四个段落进行格式化. 如果想要对文件内的所有段落进行格式化, 就执行
\texttt{1gqG}.

前面的介绍的格式化命令不仅对普通文本有效, 对其他任意类型的内容同样有效, 而且用
户可以决定使用哪种格式化函数.

用户可以为指定的文件类型设置任意的格式化函数, 方法是设置 Vim 的选项
\texttt{formatexpr}. 例如, 如果用户想要对 C 源代码进行格式化, 只需要在
\texttt{vimrc} 中添加:
\begin{vimcode}
:set formatexpr=c#Formatter()
\end{vimcode}
这行命令告诉 Vim, 在打开一个 C 源代码文件时, 使用函数 \texttt{Formatter()},
这个函数定义在 Vim 为 C 文件自动加载的文件中.

\begin{warning}
    自动回载的文件可以在 \texttt{VIMHOME} 的子目录 \texttt{autoload} 中找到.
    文件根据所服务的文件类型来命令, 以后缀 \texttt{.vim} 结束. 例如, 为 C 文
    件自动加载的文件是 \texttt{VIMHOME/autoload/c.vim}.
\end{warning}

一个格式化函数含有三个变量, 利用这三个变量可以找到待格式化的文本.
\begin{itemize}
    \item \texttt{v:num}: 待格式化的第一行的行号
    \item \texttt{v:count}: 待格式化的行的数量
    \item \texttt{v:char}: 这个变量包含了将被插入的字符, 可以为空
\end{itemize}

格式化函数的一个简单示例是:
\marginpar{124}
\begin{vimcode}
function! MyFormatter()
   let first = v:num
   let last = v:num + v:count
    while(first<=last)
       call setline(first, '> '. getline(first))
       let first = first+1
    endwhile
endfunction
\end{vimcode}
这个格式化文本接收所有的行, 在每一行的开始添加 \texttt{>}, 就像 E-mail 中的引
用.

上面展示的格式化函数非常简单, 如果需要更高级一点的, 那么函数的复杂度就会快速
增加, 因此公开可用的格式化函数非常有限 (这些函数是为了某些特定的目标而开发的).

\subsection{对齐文本}
\label{subsec:aligning_text}

在大多数字处理程序中, 最基本的格式化选项之一是左对齐, 右对齐, 或居中. 其中
一些程序甚至可以让文本两端对齐, 这样做可以让每一行的结束尽可能地向边缘靠近.

虽然上面提到的格式化类型对字处理程序来说非常常见, 但对于普通文本编辑器来说就
很少见了, Vim 就位列其中.

Vim 支持三种类型的对齐 --- 左对齐, 右对齐, 居中. 在我们介绍它们如何工作之前,
先简单解释一下支持这种对齐类型的文本编辑器为什么很少.

对于常见的文本编辑器来说, 它们不含有隐藏信息, 也就是说, 用户所能看到的, 就是
用户所能拥有的 --- 没有页面宽度, 没有对齐, 什么都没有.

另一方面, 在字处理程序中, 文档隐藏了相当丰富的信息, 这些信息告诉编辑器如何
根据用户的需要, 对文本进行格式化.

因为上面所说的情况对 Vim 并不适用, 所以用户得向 Vim 提供一些信息, 来帮助 Vim
进行格式化. 例如, 为了让编辑器知道对齐时所需的边缘, 用户需要设置文本的宽度.

说完了这些, 先来看一下居中排列的命令:
\begin{vimcode}
:[range]center WIDTH
\end{vimcode}
\marginpar{125}
命令中的 \texttt{range} 是用户希望居中的行的范围, \texttt{WIDTH} 是每行最多的
字符数. 典型的用法是在可视模式下选中待居中的文本 (按住 \key{Shift+v}, 然后移
动光标), 然后再输入命令. 按下 \texttt{:} 后, 用户将会发现 Vim 自动地把选中的
文本的范围记成 \texttt{'<,'>}, 这表示从选中文本的第一行到最后一行.

接下来, 用户需要输入 \texttt{center} 与文本的宽度. 如果选项 \texttt{textwidth}
已经设置妥当, 那就不需要输入 \texttt{WIDTH}.

如果选项 \texttt{textwidth} 的值为 0, 用户又没有在命令中设置 \texttt{WIDTH},
那么 Vim 就默认使用 80 个字符的宽度. 在字处理程序中, 用户很容易就可以看出文本
是否是居中的, Vim 对文本进行居中时, 只是在前面加上适当数量的空格, 这样
做意味着无论是在什么时候修改了文本, 用户都要重新对文本进行居中.

下面的命令用于左对齐:
\begin{vimcode}
:[range]left INDENT
\end{vimcode}
同样, 执行这个命令时需要提供行的范围, 如果需要的话, 再提供缩进的宽度. 这个命令
可以用来精确地设置文本的左边边缘.

最后是右对齐命令. 同样的, Vim 可能并不知道一行的宽度, 因此执行命令时还要提供
宽度信息. 命令的形式是:
\begin{vimcode}
:[range]right WIDTH
\end{vimcode}
范围内的行用空格缩进, 使得每一行的结束都是对齐的, 宽度由用户指定. 和居中一样,
无论何时修改了文本, 都要重新对文本进行右对齐.

\subsection{标记标题}
\label{subsec:marking_headlines}

使用普通的文本编辑器编写文档时, 用户可能需要创建自定义的格式或标记, 使得文本
更容易阅读.

为了提高可读性, 一个常见的操作是为那些用作标题的字符串作标记, 这些标题可以是
文本的章节.
\marginpar{126}

如果是字处理程序, 只需要把字体设置得更大更粗就可以了, 但是这样的字体设置对
Vim 却是不可能实现的, 因为 Vim 字体的大小是固定的. 因此, Vim 用户必须通过其他
的方法来标记标题.

笔者个人的选择是在标题的下面添加一行, 来表示这是一个标题行.

一个简单的例子是:
\begin{vimcode}
My Headline
===========
This is the text on the document. It could cantain one
or more lines of text.
\end{vimcode}
不同级别的标题可以用不同类型的标记来表示:
\begin{vimcode}
Level1
======
Level2
------
-Level3-
\end{vimcode}

为了更方便地添加下划线, 在 Vim 中可以用宏来实现, 使用这种方法不用担心下划线添
加得太多或太少.

前面两个级别的标题宏可以实现成:
\begin{vimcode}
yypVr=o
\end{vimcode}
宏的各个部分的具体涵义是:
\begin{itemize}
    \item \texttt{yy}: 复制当前行
    \item \texttt{p}: 粘贴
    \item \texttt{V}: 选中一整行
    \item \texttt{r}: 用后面的字符 (在这里是 \texttt{=}) 替换掉选中的字符
    \item \texttt{o}: 在光标的下面添加一个空行, 把光标移动到这行的开始, 并切
        换到插入模式
\end{itemize}

这个宏的基本功能是获取当前行 (标题行), 并复制它. 然后把复制得到的行的所有字符
替换成某个字符 (在这个例子中是 \texttt{-} 或 \texttt{=}), 最后, 插入一个新行,
并切换到插入模式.

对第三级别的标题来说, 我们必须采取其他办法, 不过所做的操作无非是在一行的开始
和结束添加一个连字符, 可以用一条替换命令来完成:
\marginpar{127}
\begin{vimcode}
:s/\.(.*\)/-\1-/
\end{vimcode}
我们把命令拆成三部分来说明:
\begin{itemize}
    \item \texttt{:s///}: 替换命令
    \item \verb'\(.*\)': 正则表达式, 表达式把当前行的所有字符作为输入, 并把
        它们作为搜索模式
    \item \verb'-\1-': 替换模式. 替换模式告诉 Vim 在第一个被匹配的子模式前
        添加一个连字符, 在子模式的后面再加上一个连字符
\end{itemize}

记住, 这些宏可能会写得很复杂, 但我们可以很轻易地为它们绑定一个快捷键, 例如:
\begin{vimcode}
:map h1 yypVr=0
:map h2 yypVr-0
:map h3 :s/\(.*\)/-\1-/
\end{vimcode}
现在, 用户可以在普通模式下, 通过按 \texttt{h1}, \texttt{h2}, \texttt{h3} 来添
适当的标题行. 如果用户不想在标题行的下面添加一个空行并切换到插入模式, 就把上
面的映射命令中的 \texttt{o} 删掉.

\subsection{创建列表}
\label{subsec:creating_lists}

项目列表与编号列表是文档中的常见结构. 在这一节, 我们将会介绍在 Vim 中如何方便
地创建这些列表.

让我们首先来看一个函数, 这个函数接收某个范围内被选中的行, 然后把它们转化成项
目列表. 在这个列子里, 一个简单的项目列表是:
\begin{vimcode}
* first item
* second item
* third item
\end{vimcode}
为每一行的开始添加星号的函数可以是:
\begin{vimcode}
function! BulletList()
    let lineno = line(".")
    call setline(lineno, "    * " . getline(lineno))
endfunction
\end{vimcode}
\marginpar{128}

认真看一下函数的实现, 读者就会发现它所做的不过是获取当前行, 然后再用它的一份
拷贝替换掉自身, 在前面添加一个空格, 一个星号, 然后是一个制表符.

很显然, 函数只对一行有效, 然而, 如果用户选中了一个范围内的所有行, 那么 Vim 就
会为每一个选中的行调用一次函数, 从第一行, 一直到最后一行. 如果是为每一行添加
相同的内容, 那么这种方法就很方便.

然而, 上面介绍的方法并不适合编号列表, 因为用户必须记住下一个号码从多少开始.

所以, 现在来看下面的函数, 它把某个范围内选中的行转换成编号列表 --- 列表的每一
项都占据一行:
\begin{vimcode}
function! NumberList() range
  " set line numbers in front of lines
  let beginning=line("'<")
  let ending= line("'>")
  let difsize = ending-beginning +1
  let pre = ' '
  while (beginning <= ending)
      if match(difsize, '^9*$') == 0
          let pre = pre . ' '
      endif
    call setline(ending, pre . difsize . "\t" . getline(ending))
    let ending=ending-1
    let difsize=difsize-1
  endwhile
endfunction
\end{vimcode}

这个函数稍微有点复杂, 但和它所解决的问题比起来, 还是比较简单的 --- 在每一行
的开始添加一个数字.

除此之外, 函数还做了一个额外的工作 --- 数字右对齐:
\begin{vimcode}
    1 item1
    2 item2
    ...
   10 item10
   11 item11
    ...
  100 item100
    ...
\end{vimcode}
\marginpar{129}

为了做到这样的对齐, 需要考虑两个问题:
\begin{itemize}
    \item 必须知道列表最大的编号
    \item 必须一次处理所有的行
\end{itemize}

为了解决第一个问题, 必须查看范围内第一行与最后一行的行号, 两个行号之间的差就
是行的个数, 于是我们就有了列表编号的最大值. 由于我们考虑了第二个问题, 所以这
是唯一可能的解决办法, 如果不这样做, 那我们一次只能处理一行.\footnote{This is
only possible because the second issue is also taken care of, without which
the function would only know about the current line.}

解决第二个问题的办法是在函数名的后面添加关键字 \texttt{range}, 这个关键字告诉
Vim, 函数是对一个范围内的行作操作, 而不仅仅是一行.

函数从范围内的最后一行处理到第一行, 无论何时碰到一个只含有 9 的数字 (比如 99,
9999), 都说明编号中的字符少了一个 (例如, 从第 1000 行到第 999 行), 为了弥补缺
少的一个字符, 函数为缩进新增一个空格. 通过这种办法, 可以一直保证数字的右对齐,
无论范围内有多少行.

\begin{warning}
    在 \url{http://www.vimoutliner.org}, 用户可以找到利用标题格式化, 列表格式
    化等技术来为文档列提纲的脚本. 如果用户需要在 Vim 中为文档列提纲, 最好试一
    下这个脚本.
\end{warning}

\section{格式化代码}
\label{sec:formatting_code}

在对代码进行格式化时经常需要考虑很多因素. 每一种语言都有自己的语法规则, 还有
些语言对格式非常依赖. 在有些情况下, 程序员需要按照公司给出的规定来格式化代码.

那么, Vim 如何知道用户想把代码格式化成什么样子? 简单来说 Vim 并不需要知道. 但
是 Vim 有办法让用户完全按照自己的需要来格式化代码.
\marginpar{130}

虽然具体的格式化细节不尽相同, 但是它们都遵循一些基本的规则, 也就是说, 用户只
需要关心不同的地方即可. 在大部分情况下, 格式化规则的修改可以通过一系列的 Vim
选项来设置, 在这些选项当中, 比较重要的有以下几个:
\begin{itemize}
    \item \texttt{formatoptions}: 这个选项负责特定格式的设置 (见 \texttt{:help
        'fo'})
    \item \texttt{comments}: 什么是注释, 以及如何对它们进行格式化 (见
        \texttt{:help 'com'})
    \item \texttt{(no)expandtab}: 用空格代替制表符 (见 \texttt{:help
        'expandtab'})
    \item \texttt{softtabstop}: 一个制表符可以用多少个空格来代替 (见
        \texttt{:help 'sts'})
    \item \texttt{tabstop}: 一个制表符的宽度 (见 \texttt{:help 'ts'})
\end{itemize}

通过这些选项, 用户几乎可以设置与缩进相关的方方面面. 但是光有这些还不足够,
用户仍
然需要告诉 Vim 是否需要自动缩进, 还是由用户手动完成缩进. 如果用户希望由 Vim
来完成缩进, 可以通过 4 种方法来完成, 在下面的小节里, 我们将会介绍与代码缩进
相关的一系列选项.

\subsection{Autoindent}
\label{subsec:autoindent}

Autoindent 是缩进代码最简单的方式. 它的功能仅仅是与上一行保持相同的缩进层次.
所以说, 如果当前行缩进了 4 个空格, 按下 \key{Enter} 后插入的空行也会缩进 4 个
空格. 至于什么时候修改缩进的层次则完成取决于用户. 这种缩进方式适用于若干行需
要保持相同缩进层次的语言. 打开 autoindent 的命令是 \texttt{:set autoindent}
或 \texttt{:set ai}.

\subsection{Smartindent}
\label{subsec:smartindent}

Smartindent 比 autoindent 稍微智能一些. 它仍然可以让上一行的缩进层次保持到下
一行, 但用户无需手动修改缩进层次. Smartindent 可以识别出 C 语言的大部分结构,
并根据它们来决定何时增加/减少缩进层次. 由于许多编程语言都或多或少地都继承了 C
语言的语法, 所以 smartindent 也可以应用到其他语言. 打开 smartindent 可以用下
面两个命令中的任意一个:
\begin{itemize}
    \item \texttt{:set smartindent}
    \item \texttt{:set si}
\end{itemize}
\marginpar{131}

\subsection{cindent}
\label{subsec:cindent}

Cindent 经常被称为 clever indent(聪明的缩进) 或 configurable indent(可配置的
缩进), 这是因为与前面介绍的两种缩进相比, 它的可配置性更强. 有三种设置选项:
\begin{itemize}
    \item \texttt{cinkeys}: 这个选项包含了一个列表, 列表中的各项之间用逗号分
        开, Vim 可以根据列表中的项来改变缩进层次. 一个例子是: \texttt{:set
            cinkeys="0\{,0\},0\#,:"}, 意思是说无论何时碰到一个以 \verb'{', 或
            \verb'}', 或 \verb'#' 作为开始的行, 或者是以 \verb':' 作为结束的
            行 (很多语言的 \texttt{switch} 结构都用到了 \verb':'), Vim 都要
            再缩进一层. \texttt{cinkeys} 的默认值是 \texttt{"0\{, 0\}, 0), :,
        0\#, !\textasciicircum F, o, O, e"}, 更多的信息见 
        texttt{:help cinkeys}.
    \item \texttt{cinoptions}: 这个选项包含了所有的, 专门用于 cindent 的选项.
        它是一个各项之间由逗号分开的列表, 通过这个列表可以设置大量的选项. 一
        个例子是 \texttt{cinoptions=">2,\{3,\}3"}, 意思是说在正常的缩进之上,
        再额外添加两个空格, 另外, 在 \verb'{' 与 \verb'}' 的左边添加三个空格,
        以便与前一行作比较. 因此, 如果目前的缩进是 4 个空格, 那么刚才所说的
        设置会使代码变成 (代码中的点表示一个空格):
        \begin{vimcode}
    if (a == b)
    ...{
    ......print "hello";
    ...}
        \end{vimcode}
        % FIXME
        \texttt{cinoptions} 的默认值是 \texttt{">s,e0,n0,f0,\{0,\}0,%
        \textasciicircum 0,%
        :s,=s,l0,b0,gs,hs,ps,ts,is,+s,c3,C0,/0,(2s,us,U0,w0,W0,m0,j0,%
        )20,*30"}, 更多的信息见 \texttt{:help 'cinoptions'}.
    \item \texttt{cinwords}: 这个选项包含的关键词会让 Vim 在下一行增加缩进. 一
        % FIXME
        个例子是 \texttt{:set cinwords="if,else,do,while,for,switch"}, 这同时
        也是它的默认值. 更多的信息见 \texttt{:help 'cinwords'}.
\end{itemize}
\marginpar{132}

\subsection{Indentexpr}
\label{subsec:indentexpr}

Indentexpr 是最灵活的缩进选项, 但同时也是最复杂的. 使用 indentexpr 时, 它会
对一个表达式求值, 然后计算出一行的缩进. 因此, 用户写出的表达式必须能被 Vim
求值. 打开这个选项的方法是为它设置一个表达式, 比如:
\begin{vimcode}
:set indentexpr=MyIndenter()
\end{vimcode}
命令中的 \texttt{MyIndenter()} 是一个函数, 负责计算行的缩进.

函数的一个简单例子是模仿选项 autoindent 的功能:
\begin{vimcode}
function! MyIndenter()
   " Find previous line and get it's indentation
   let prev_lineno = s:prevnonblank( v:lnum)
   let ind = indent( prev_lineno )
   return ind
endfunction
\end{vimcode}

即使是想把函数的功能增强一点, 函数的复杂度也会迅速增长. Vim 自带了多种编程语言
的缩进表达式, 如果用户想要自己编写缩进表达式, 可以以它们为基础, 再加以修改. 用
户可以在 \texttt{VIMHOME} 的子目录 \texttt{indent} 中找到实现代码.

更多的信息见 \texttt{:help 'indentexpr'} 与 \texttt{:help 'indent-expression'}.

\subsection{代码块快速格式化}
\label{subsec:fast_code_block_formatting}

缩进选项设置完毕后, 用户可能想根据新设置的选项更新一下代码的格式. 为了更新代码,
只需要告诉 Vim 重新缩进第一行到最后一行, 具体的命令是:
\begin{vimcode}
1G=G
\end{vimcode}
如果把命令拆开来看, 各个部分的意思是:
\begin{itemize}
    \item \texttt{1G}: 跳到文件的第一行 (也可以使用 \texttt{gg})
    \item \texttt{=}: 根据格式化的配置对文本加以缩进
    \item \texttt{G}: 跳到文件的最后一行 (缩进的结束位置)
\end{itemize}
\marginpar{133}

如果把命令映射到按键上, 使用起来就更加方便:
\begin{vimcode}
:nmap <F11> 1G=G
:imap <F11> <ESC>1G=Ga
\end{vimcode}
后一个命令尾巴的 \texttt{a} 是为了在格式化完成后, 重新回到插入模式, 因为在执行
命令前就处在插入模式下. 现在, 只需按下 \key{F11}, 就可以对整个缓冲区的文本重新
缩进.

\begin{warning}
    注意, 如果待缩进的代码含有编程错误, 比如在 C 代码中漏写了一个分号, 那么代
    码就不会被正确地缩进. 利用这个性质可以在一定程序上检查编程错误.
\end{warning}

有时候, 用户可能仅仅是想格式化某一小块代码, 对于这种情况, 用户可以使用代码块
自然形成的范围, 或者是在可视模式下选择一段代码, 然后再格式化.

后一种方法比较简单. 先是切换到可视模式, 比如按下 \key{Shift+v}, 然后再按
\key{=} 来重新缩进被选中的代码行.

另一方面, 如果要使用代码块, 则有多种不同的方法来实现. 在选择代码块时, Vim 提供
了很多的方法. 为了与一个缩进代码的命令作组合, 我们必须介绍这些不同类型的代码
块, 以及选择它们的命令:

\begin{itemize}
    \item \verb'i{': 内部块 (Inner block), 指的是 \verb'{' 与 \verb'}'
        之间的所有内容 (
        不包括 \verb'{' 与 \verb'}'). 还可以用 \verb'i}' 与 \texttt{iB} 来选
        择.
    \item \verb'a{': 一个块 (A block), 指的是 \verb'{' 与 \verb'}'
        之间的所有内容 (包括
        \verb'{' 与 \verb'}'). 还可以用 \verb'a}' 与 \verb'aB' 来选择.
    \item \verb'i(': 括号内 (Inner parenthesis), 指的是 \verb'(' 与 \verb')'
        之间的所有内容 (不包
        括 \verb'(' 与 \verb')'), 还可以用 \verb'i)' 与 \verb'ib' 来选择.
    \item \verb'a(': 一对括号 (A parentheses), 指的是 \verb'(' 与 \verb')'
        之间的所有内容 (包括 \verb'(' 与 \verb')'), 还可以用 \verb'a)' 与
        \verb'ab' 来选择.
    \item \verb'i<': 内部 \verb'<>' 块 (Inner \verb'<>' block), 指的是
        \verb'<' 与 \verb'>'
        之间的所有内容 (不包括 \verb'<' 与 \verb'>'), 还可以用 \verb'i>'
        来选择.
    \item \verb'a<': 一个 \verb'<>' 块 (A \verb'<>' block), 指的是
        \verb'<' 与 \verb'>'
        之间的所有内容 (包括 \verb'<' 与 \verb'>'), 还可以用 \verb'a>'
        来选择.
    \item \verb'i[': 内部 \verb'[]' 块 (Inner \verb'[]' block), 指的是
        \verb'[' 与 \verb']'
        之间的所有内容 (不包括 \verb'[' 与 \verb']'), 还可以用 \verb'i]'
        来选择.
    \item \verb'a[': 一个 \verb'[]' 块 (A \verb'[]' block), 指的是
        \verb'[' 与 \verb']'
        之间的所有内容 (包括 \verb'[' 与 \verb']'), 还可以用 \verb'a]'
        来选择.
\end{itemize}
\marginpar{134}

现在, 我们已经知道了 Vim 是如何看待一个代码块的, 接下来, 我们只需要告诉它如何
操作代码块. 在我们的例子中, 是希望 Vim 重新缩进代码块, 前面已经说到,
\texttt{=} 可以用来重新缩进代码, 所以重新缩进代码块的一个示例是:
\begin{vimcode}
=i{
\end{vimcode}
现在, 让我们用上面的命令重新缩进下面的代码 (\verb'|' 表示光标的当前位置):
\begin{vimcode}
if( a == b )
  {
    print |"a equals b";
  }
\end{vimcode}
执行命令后, 代码变成 (默认使用 C 语言的缩进格式):
\begin{vimcode}
if( a == b )
  {
    print |"a equals b";
  }
\end{vimcode}
如果改用 \verb'a{' 来选择代码块, 代码就会变成:
\begin{vimcode}
if( a == b )
  {
    print "a equals b";
  }
\end{vimcode}
在最后一段代码中, 命令 \verb'=a{' 同时纠正了花括号与打印语句的缩进.

有时候, 用户可能会遇到一个带有多级代码块的代码块, 并且用户希望重新缩进当前代码
块及包围它的代码块. 不用担心, Vim 提供了一个很方便的做法. 比如, 如果用户想要重
新缩进当前代码块及包围它的代码块, 只需要把光标放在相对内层的代码块上, 并执行:
\begin{vimcode}
=2i{
\end{vimcode}
这个命令告诉 Vim 重新缩进内层代码块中的两层, 从当前 ``活跃'' 的块到该块的外层.
可以把命令中的 \texttt{2} 替换成用户想要重新缩进的层数. 当然, 还可以把
\texttt{i} 替换成其他的代码块选择命令, 从而精确地选择待缩进的代码块.

上面介绍的这些, 就是缩进代码所要掌握的全部内容.
\marginpar{135}

\subsection{自动格式化粘贴的代码}
\label{subsec:auto_format_pasted_code}

经常有程序员告诉我们, 他们会经常复用已有的代码, 这意味着他们经常需要复制与粘
贴代码.

许多用户在把代码粘贴到 Vim 窗口中时, 都会碰到过 ``阶梯效应'' --- 当插入代码时,
Vim 会尝试对代码进行缩进, 造成的结果是越往后, 行的缩进层次越深, 类似于阶梯:
\begin{vimcode}
code line 1
    code line 2
        code line 3
            code line 4
               ...
\end{vimcode}
解决这个问题的办法通常是把 Vim 设置成 粘贴 (paste) 模式, 用到的命令是:
\begin{vimcode}
:set paste
\end{vimcode}
代码粘贴完毕后, 用下面的命令把 Vim 设置回正常的插入模式:
\begin{vimcode}
:set nopaste
\end{vimcode}

那么, 除此之外, 是否还有其他的方法呢? Vim 是否可以根据文件中已有的其他代码来
自动格式化被粘贴的代码? 其实很简单.
\begin{vimcode}
p=`]
\end{vimcode}
上面的命令只是把普通的粘贴命令 (\texttt{p}) 和另一命令组合, 后者会自动缩进前
面插入的行 (\texttt{=`]}). 这个命令基于下面的事实: 当用户使用命令 \texttt{p}
来粘贴文本时, 光标将会停留在被粘贴文本的第一个字符上, 组合上命令 \texttt{`]}
之后, 光标就会移动到最近一次插入的文本行的最后一个字符上, 并且可以从被粘贴文
本的第一行移动到最后一行.

所以, 用户所要做的就是把这个命令绑定到一个快捷键上, 并用该快捷键来粘贴文本.
\marginpar{136}

\section{使用外部格式化工具}
\label{sec:using_external_formatting_tools}

即使是经验丰富的 Vim 用户也会经常说 Vim 可以做任何事情, 虽然事实上并不是这样,
但其实已经很接近了 --- 对于 Vim 不能做的事, 最好借助外部工具来完成.

在下面的小节里, 我们将会介绍最经常使用的格式化工具, 以及如何使用它们.

\subsection{Indent}
\label{subsec:indent}

程序 Indent 可能是最常使用的 Vim 外部工具. 从 80 年代开始, 它就被安装到了各种
不同的 Unix 平台中, 后来还被移值到了 Microsoft Windows.

如同它的名字那样, 这个程序的功能就是缩进代码 --- 特别是那些类 C 代码. 用户可
会感到奇怪 --- 既然 Vim 本来就可以缩进代码, 而且做得也很不错, 那干嘛还要用外
部工具呢? 这个问题非常好, 答案是 Vim 确实在这一方面做得很好, 但是 Indent 做得
更好, 并且更容易在多个编辑器之间对缩进进行标准化.

由于 indent 专门用于缩进代码, 所以它的效率和 Vim 相比会高出很多, 对于后者来
说, 缩进代码只是一项特性而已, 而不是唯一的特性. Indent 对代码的理解力很高, 可
以根据代码来缩进 --- 即使含有语法错误.

那么, 在 Vim 中应该如何使用 Indent? 在前面的几节, 我们已经介绍了 Vim 关于缩
进的几个选项, 然而, 现在只需要一个选项即可:
\begin{vimcode}
:set equalprg=PROGRAM
\end{vimcode}

这个选项的功能是当使用命令 \texttt{=} 缩进代码时, Vim 应该使用哪种外部工具.
对于
Indent, 只要把 \texttt{PROGRAM} 替换成程序 Indent 的路径即可. 设置之后, 无论
何时使用缩进命令 (比如 \texttt{1G=G}), Vim 就会把待缩进的文本输送给程序
Indent. 甚至可以在选项中加上命令行选项.

Indent 有着非常丰富的命令行选项, 另外还可以通过它的配置文件来修改 Indent 的
行为.
\marginpar{137}

\begin{warning}
    可以从以下网址下载到最新版的 Indent:
    \url{http://www.gnu.org/software/indent/}.
\end{warning}

\subsection{Berkeley Par}
\label{subsec:berkeley_par}

90 年代早期, Adam M. Costello 开始开发一个简单的命令行工具, 这个工具的唯一功
能是按照用户的要求, 重新格式化文本中的段落. 工具的名字是 Par, 一两年后, 该工
具拥有了非常丰富的功能, 几乎可以重新格式化任意类型的段落.

正因为如此, Par 成了 Vim 的理想工具. 现在让我们来看一些使用例子.

假如, 用户想重新格式化段落, 使得每一行最大长度不超过 78 个字符, 那就在 Vim 中
执行:
\begin{vimcode}
:set formatprg=par\ -w78
\end{vimcode}

选项 \texttt{formatprg} 告诉 Vim, 当使用命令 \texttt{gq} 时, 应该调用哪个外部
工具来格式化文本. 值得注意的是, 程序和它的参数之间的空格用一个反斜杆转义, 这
做就可以让 Vim 把程序与它的参数看作一个整体.

\begin{warning}
    注意, 如果 \texttt{formatexpr} 为空, 则只会使用 \texttt{formatprg}, 否则
    的话, 就使用 \texttt{formatexpr}.
\end{warning}

在前面我们曾经提到过, Vim 无法对文本进行两端对齐, 幸运的是, Par 可以帮助 Vim
完成这件工作. 只需要在上面的命令行参数中加上 \texttt{j} (意思是 justify),
就可完成两端对齐:
\begin{vimcode}
:set formatprg=par\ -w78j
\end{vimcode}

Par 不仅可以用在普通文本中, 对于代码的某些部分也可以使用, 比如注释.

假设用户拥有下面一段注释:
\begin{vimcode}
/********************************************************************/
/* This function helps you modify a string and remove all */
/* unnecessary characters . */
/* Don't use this on widechar strings or strings shorter than 10 */
/* characters */
/********************************************************************/
\end{vimcode}
\marginpar{138}
用户可以选中这些注释, 然后执行:
\begin{vimcode}
!par 60r
\end{vimcode}
(Vim 会在 \texttt{!} 的前面自动加上 \texttt{'<,'>})

执行后, 注释变成:
\begin{vimcode}
/**********************************************************/
/* This function helps you modify a string and remove all */
/* unnecessary characters . Don't use this on widechar    */
/* strings or strings shorter than 10 characters          */
/**********************************************************/
\end{vimcode}

只需要一个简单的命令, 用户就可以把丑陋的, 未格式化过的注释变成精心排列的文本.

Par 的手册页包含了许多例子.

% TODO
% You could easily map different Par commands to different keys in Vim and this
% way have formatting keys for all text, comments, lists, and so on.

\subsection{Tidy}
\label{subsec:tidy}

如果用户从事网页开发, 或者 XML 文件编辑, 那么程序 tidy 就会成为你的好帮手. 这
个程序可以清理代码, 使之符合 W3C 规范. 符合 W3C 规范指的是代码的构造必须遵守
由 W3C 规定的 HTML 守则.

作为一个网页开发人员, 笔者有时候会浏览别人写的 HTML 或 XML 文件, 打开后只看
到了一团乱麻. 后来, 笔者会在打开之前, 先用 tidy 处理一下所有的 \texttt{.xml},
\texttt{.html} 或 \texttt{.html} 文件. 这个工作可以借助 Vim 的自动命令来完成,
自动命令可以添加在文件 \texttt{vimrc} 中.

对于 XML 文件, 自动命令的具体代码是:
\begin{vimcode}
au FileType xml exe ":silent 1,$!tidy --input-xml true --indent yes -q"
\end{vimcode}
对于 HTML 文件, 自动命令是:
\begin{vimcode}
au FileType html exe ":silent 1,$!tidy --indent yes -q"
\end{vimcode}
\marginpar{139}
需要注意得是, 这个命令会在打开文件时悄无声息地修改掉文件内容. 在上面两个命令
中, Vim 会在指定的路径中搜索程序 tidy, 无论是 Linux 还是 Windows.

观察 tidy 的命令行参数后可以看到, 它也可以用来重新缩进 HTML/XML. 这个选项提高
了文件的可读性, 也更容易获取文件的概览.

因为 tidy 可以检查出文档的错误, 所以用户可以把 tidy 的命令映射到一个键上, 这
样的话就可以随时检查文档是否符合 W3C 规范.

\begin{warning}
    可以到 \url{http://tidy.sourceforge.net} 下载到最新版的 tidy.
\end{warning}

\section{小结}
\label{sec:advanced_formatting_summary}

在这一章, 我们介绍了如何更好地格式化普通文本与代码.

首先, 我们介绍了如何通过一对简单的 Vim 命令, 把文本格式化成更易阅读的形式.
我们还介绍了文本的对齐, 还解释了纯文本编辑器在对齐方面的困难. 接下来, 我们创建
了一个函数, 该函数用于标记标题行, 以及生成无编号与编号列表. 我们还看到了 Vim
非常的灵活, 比如用户可以告诉 Vim 是否需要对选中的文本中的每一行执行操作, 又或
者是是否让用户来处理, 并将待处理的行一次都反馈给用户的函数.

接下来, 我们介绍了在 Vim 中如何格式化代码, 尤其是缩进. 因为每个人都有
着自己独特的编码风格, 所以很难有一个通用的功能来完成代码的格式化. Vim 提供了
一个灵活的接口, 使得用户可以按照自己的需要来设置. 我们还介绍了一些用来迅速格
式化代码的技巧, 甚至还包括如何格式化从其他地方粘贴到 Vim 中的代码.

最后, 我们介绍了如何利用外部工具来使得 Vim 更加完美. 外部工具可以帮助用户格式
化文本与代码, 我们介绍了几种比较流行的外部工具, 以及在 Vim 中使用它们的方法.
\marginpar{140}

到了这里, 用户应该可以熟练地利用 Vim 内建功能, 使得它更符合自己的需求. 接下来,
我们将学习如何编写 Vim 脚本.
% end of chapter 5
