% vim: ts=4 sts=4 sw=4 tw=80
\chapter{高级 Vim 脚本编程}
\label{chap:extended_vim_scripting}
\marginpar{175}

在前面一章, 我们已经学习了开发 Vim 脚本的基础知识, 现在, 我们将要把前面所学的
知识融会贯通, 按照结构化的方法把它们组织起来, 并对脚本进行测试.

这一章涵盖的主题包括:
\begin{itemize}
    \item 如何组织 Vim 脚本的结构
    \item Vim 脚本开发的一些技巧
    \item 如何调试 Vim 脚本
    \item 如何在 Vim 脚本中使用其他脚本语言
\end{itemize}

阅读完这一章之后, 读者将有能力运用 Vim 脚本语言与其他脚本语言开发出自己的脚本.
也就是说, 读者将有能力扩展 Vim 的功能.

\section{脚本结构}
\label{sec:script_structure}

前面我们已经介绍了 Vim 脚本的各个要素, 现在我们需要知道如何把它们组织在一起,
从而形成一个完成的脚本.

在大部分情况下, Vim 脚本仅由单个文件组成, 因此这一间的示例也仅限于单个文件. 我
们还打算让其他人能够获取到脚本, 因此需要保证代码的可读性.

在下面的几节里, 我们将会逐个介绍一个格式良好的脚本的各个要素.
\marginpar{176}

\subsection{脚本头部}
\label{subsec:script_header}

一个脚本文件在开头最好配上一个头部信息, 用来写明该脚本的用途. 头部应该包含下
面这些信息:
\begin{itemize}
    \item 脚本的维护人员
    \item 最后一次更新的版本
    \item 发布许可证 (最重要的信息)
\end{itemize}
一个示例是:
\begin{vimcode}
    " myscript.vim  : Example script to show how a script is structured.
    " Version       : 1.0.5
    " Maintainer    : Kim Schulz<kim@schulz.dk>
    " Last modified : 01/01/2007
    " License       : This script is released under the Vim License.
\end{vimcode}
注意头部信息都是以 \texttt{"} 开始的, 也就是说它们都是一些注释.

头部还可以包含其他信息, 比如脚本可能依赖于其他脚本, 或者是该脚本对 Vim 版本
的要求.

\subsection{脚本加载检查}
\label{subsec:script_loaded_check}

一种良好的编程习惯是检查脚本是否已被加载, 如果是, 则在继续其他操作之前, 先执
行卸载操作.
% It is always a good practice to check if the script has already been loaded
% once, and if it has, then unload functions before moving on.
这是因为脚本不仅会安装在系统的全局目录中, 还会安装在用户的
\texttt{VIMHOME} 目录下.

检查脚本是否已加载的函数可以这样实现:
\begin{vimcode}
if exists("loaded_myscript")
    finish "stop loading the script
endif
let loaded_myscript = 1
\end{vimcode}
如果脚本未被加载, \texttt{if} 条件判断为假, 因此函数设置变量
\texttt{loaded\_myscript}.

当下一次加载脚本时, \texttt{if} 条件判断为真, 因为此时变量
\texttt{loaded\_myscript} 已经存在, 然后函数停止加载脚本.
\marginpar{177}

在某些情况下, 停止加载脚本可能并不是最好的做法, 因为用户可以修改了
\texttt{VIMHOME} 目录中的脚本版本. 所以, 这时候应该先卸载脚本, 然后再重新加载
脚本. 完成这项功能的函数可以这样实现:
\begin{vimcode}
if exists("loaded_myscript")
    delfunction MyglobalfunctionB
    delfunction MyglobalFunctionC
endif
let loaded_myscript = 1
\end{vimcode}

脚本开发人员无法知道 Vim/vi 当前是否处于兼容模式下 (如果是 vi 的话, 则比较有可
能), 所以比较好的做法是在脚本中保存编辑器的兼容模式. 这样做就可以确保脚本可以
正常地使用 Vim 特定的功能. 把下面的代码加到加载检查语句的后面:
\begin{vimcode}
:let s:global cpo = &cpo "store current compatible-mode
                         " in local variable
:set cpo&vim             " go into nocompatible-mode
\end{vimcode}
最后在脚本的末尾恢复原来的兼容模式:
\begin{vimcode}
:let &cpo = s:global_cpo
\end{vimcode}

\subsection{脚本配置}
\label{subsec:script_configuration}

用户在阅读别人开发的脚本时, 总是从头开始看起, 所以我们最好把所有的配置选项都
放在脚本的开始. 配置选项可以是外部程序的路径, 脚本依赖的特定文件的名字, 文件
类型等.

用户可能会在他的 \texttt{vimrc} 文件中改变 Vim 的配置, 所以开发人员需要确保
脚本不会覆盖掉他原来的配置. 方法是事先检查配置是否已被设置过, 只有在没有设置
过的情况下才设置它.

脚本中的设置语句可以这样写:
\begin{vimcode}
" variable myscript_path
if !exists("myscript_path")
    let s:vimhomepath = split(&runtimepath, ',')
    let s:myscript_path = s:vimhomepath[0]."/plugin/myscript.vim"
else
    let s:myscript_path = myscript_path
    unlet myscript_path
\end{vimcode}
\marginpar{178}
\begin{vimcode}
endif

" variable myscript_indent
if !exists("myscript_indent")
    let s:myscript_indent = 4
else
    let s:myscript_indent = myscript_indent
    unlet myscript_indent
endif
\end{vimcode}
上面的例子设置了两个配置变量 --- \texttt{myscript\_path} 与
\texttt{myscript\_indent}. 我们查找变量是否已存在, 如果不存在, 则在脚本的作用
域内设置变量的默认值 (比如, \texttt{s:myscript\_path}).

如果用户已经设置了变量, 那么变量的值就赋给脚本作用域内的同名变量.

最终, 用户定义的变量用 \texttt{unlet} 移除, 这样的话, 它就不对全局作用域产生
影响 --- 配置只需要在脚本内起作用即可.

\subsection{按键映射}
\label{subsec:key_mappings}

如果需要的话, 还可以添加按键映射, 这些映射可以是函数调用, 变量设置等. 和配置
变量一样, 在设置某个映射之前, 需要检查一下该映射是否已经建立好了, 检查的语句
可以这样写:
\begin{vimcode}
if !hasmapto('<Plug>MyscriptMyfunctionA')
    map <unique> <Leader>a <Plug>MyscriptMyfunctionA
endif
\end{vimcode}
在上面的代码中包含了一些我们已经没有介绍过的东西:
\begin{itemize}
    \item \texttt{hasmapto()}: 用于检查某个函数映射是否存在的函数
    \item \texttt{<unique>}: 如果相同的映射存在, 则报错.
    \item \texttt{<Leader>}: 由用户决定使用哪个映射前导字符. \texttt{<Leader>}
        将会被全局变量 \texttt{mapleader} 的值所替换.
    \item \texttt{<Plug>}: 为一个函数建立一个唯一的全局标识符, 这样的话, 它就
        不会与全局作用域中的其他函数产生冲突.
\end{itemize}
\marginpar{179}
把这些元素都组织在一起之后, 我们就创建一个脚本, 用来检查是否存在某个映射, 已经
绑定到唯一的函数标识 \texttt{<Plug>MyscriptMyfunctionA} 上. 如果这样的映射不存
在, 就把 \texttt{<Leader>a} 映射到标识符上 --- 除非 \texttt{<Leader>a} 已经被
其他人占用了, 此时 Vim 就会报错.

用户可能好奇 Vim 是如何从 \texttt{<Plug>MyscriptMyfunctionA} 得到
\texttt{MyfunctionA()} 的. 为此, 我们还需要建立其他一些映射:
\begin{vimcode}
noremap <unique> <script> <Plug>MyscriptMyfunctionA <SID>MyfunctionA
noremap <SID>MyfunctionA :call <SID>MyfunctionA()<CR>
\end{vimcode}
第一个命令把 \texttt{<Plug>MyscriptMyfunctionA} 映射到 \texttt{<SID>MyfunctionA}
上. 我们在代码中用到了 \texttt{<SID>}, 这是因为
