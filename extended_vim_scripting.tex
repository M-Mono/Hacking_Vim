% vim: ts=4 sts=4 sw=4 tw=80
\chapter{高级 Vim 脚本编程}
\label{chap:extended_vim_scripting}
\marginpar{175}

在前面一章, 我们已经学习了开发 Vim 脚本的基础知识, 现在, 我们将要把前面所学的
知识融会贯通, 按照结构化的方法把它们组织起来, 并对脚本进行测试.

这一章涵盖的主题包括:
\begin{itemize}
    \item 如何组织 Vim 脚本的结构
    \item Vim 脚本开发的一些技巧
    \item 如何调试 Vim 脚本
    \item 如何在 Vim 脚本中使用其他脚本语言
\end{itemize}

阅读完这一章之后, 读者将有能力运用 Vim 脚本语言与其他脚本语言开发出自己的脚本.
也就是说, 读者将有能力扩展 Vim 的功能.

\section{脚本结构}
\label{sec:script_structure}

前面我们已经介绍了 Vim 脚本的各个要素, 现在我们需要知道如何把它们组织在一起,
从而形成一个完成的脚本.

在大部分情况下, Vim 脚本仅由单个文件组成, 因此这一间的示例也仅限于单个文件. 我
们还打算让其他人能够获取到脚本, 因此需要保证代码的可读性.

在下面的几节里, 我们将会逐个介绍一个格式良好的脚本的各个要素.
\marginpar{176}

\subsection{脚本头部}
\label{subsec:script_header}

一个脚本文件在开头最好配上一个头部信息, 用来写明该脚本的用途. 头部应该包含下
面这些信息:
\begin{itemize}
    \item 脚本的维护人员
    \item 最后一次更新的版本
    \item 发布许可证 (最重要的信息)
\end{itemize}
一个示例是:
\begin{vimcode}
    " myscript.vim  : Example script to show how a script is structured.
    " Version       : 1.0.5
    " Maintainer    : Kim Schulz<kim@schulz.dk>
    " Last modified : 01/01/2007
    " License       : This script is released under the Vim License.
\end{vimcode}
注意头部信息都是以 \texttt{"} 开始的, 也就是说它们都是一些注释.

头部还可以包含其他信息, 比如脚本可能依赖于其他脚本, 或者是该脚本对 Vim 版本
的要求.
